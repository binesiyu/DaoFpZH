\documentclass[DaoFP]{subfiles}
\begin{document}

\chapter{白板}

编程始于类型和函数。你可能对类型和函数有一些先入为主的观念:请摒弃它们!它们会蒙蔽你的思维。

不要考虑事物在硬件中是如何实现的。计算机只是众多计算模型中的一种。我们不应过于依赖它。你可以在脑海中,或用纸笔进行计算。物理载体与编程的理念无关。

\section{类型与函数}

引用老子的话\footnote{老子的现代拼写是Laozi,但我会使用传统的拼写。老子是《道德经》的半传奇作者,这是一部关于道家的经典著作。}:\emph{可道之道,非常道}。换句话说,类型是一个原始概念。它无法被定义。

我们也可以将其称为\emph{对象}或\emph{命题}。这些是数学不同领域中用来描述它的术语(分别是类型理论、范畴论和逻辑学)。

可能存在多个类型,因此我们需要一种命名它们的方式。我们可以通过指向它们来命名,但由于我们希望有效地与他人交流,我们通常会给它们命名。因此,我们会谈论类型$a$、$b$、$c$;或\hask{Int}、\hask{Bool}、\hask{Double}等。这些只是名称。

类型本身没有意义。它的特殊之处在于它如何与其他类型连接。这些连接由箭头描述。一个箭头有一个类型作为其源,一个类型作为其目标。目标可以与源相同,在这种情况下,箭头会循环。

类型之间的箭头称为\emph{函数}。对象之间的箭头称为\emph{态射}。命题之间的箭头称为\emph{蕴涵}。这些只是数学不同领域中用来描述箭头的术语。你可以互换使用它们。

命题是可能为真的事物。在逻辑学中,我们将两个对象之间的箭头解释为$a$蕴涵$b$,或$b$可以从$a$推导出来。
\pagebreak

两个类型之间可能有多个箭头,因此我们需要给它们命名。例如,这里有一个名为$f$的箭头,从类型$a$指向类型$b$

\[ a \xrightarrow f b \]

一种解释方式是,函数$ f$接受一个类型为$a$的参数,并产生一个类型为$b$的结果。或者$ f$是一个证明,如果$a$为真,那么$b$也为真。

注意:类型理论、λ演算(编程的基础)、逻辑学和范畴论之间的联系被称为Curry-Howard-Lambek对应。

\section{阴阳}

对象由其连接定义。箭头是两个对象连接的证明或见证。有时没有证明,对象是断开的;有时有许多证明;有时只有一个证明——两个对象之间的唯一箭头。

\index{唯一}\emph{唯一}是什么意思?这意味着如果你能找到两个这样的箭头,那么它们必须相等。

一个对象如果对每个对象都有一个唯一的出箭头,则称为\emph{初始对象}。

它的对偶是一个对象,对每个对象都有一个唯一的入箭头。它被称为\emph{终止对象}。

在数学中,初始对象通常用$0$表示,终止对象用$1$表示。

从$0$到任何对象$a$的箭头表示为$\mbox{!`}_a$,通常简写为$\mbox{!`}$。

从任何对象$a$到$1$的箭头表示为$!_a$,通常简写为$!$。

初始对象是万物的源头。作为一个类型,它在Haskell中被称为\hask{Void}。它象征着万物从中产生的混沌。由于从\hask{Void}到万物都有一个箭头,因此从\hask{Void}到自身也有一个箭头。

\[
 \begin{tikzcd}
 \hask{Void}
 \arrow[loop]
 \end{tikzcd}
\]

因此,\hask{Void}生成了\hask{Void}和一切其他事物。

终止对象统一了一切。作为一个类型,它被称为Unit。它象征着终极秩序。

在逻辑学中,终止对象象征着终极真理,用$T$或$ \top$表示。从任何对象到它都有一个箭头的事实意味着$ \top$无论你的假设是什么都是真的。

对偶地,初始对象象征着逻辑上的假、矛盾或反事实。它被写为False,并用倒置的T $ \bot$表示。从它到任何对象都有一个箭头的事实意味着你可以从假前提中证明任何事情。

在英语中,有一种特殊的语法结构用于反事实的蕴涵。当我们说“如果愿望是马,乞丐就会骑马”时,我们的意思是愿望与马之间的等式意味着乞丐能够骑马。但我们知道前提是假的。

编程语言让我们能够相互交流并与计算机交流。有些语言更容易被计算机理解,另一些则更接近理论。我们将使用Haskell作为折衷方案。

在Haskell中,终止类型的名称是\hask{()},一对空括号,发音为Unit。这个符号稍后会变得有意义。

Haskell中有无限多的类型,并且从\hask{Void}到每个类型都有一个唯一的函数/箭头。所有这些函数都使用相同的名称:\hask{absurd}。

\begin{center}
\begin{tabular} {|c | c | c|}
\hline
编程 & 范畴论 & 逻辑学 \\
\hline
类型 & 对象 & 命题 \\
函数 & 态射(箭头) & 蕴涵 \\
\hask{Void} & 初始对象, $0$ & False $\bot$ \\
\hask{()} & 终止对象, $1$ & True $\top$ \\
\hline

\end{tabular}
\end{center}

\section{元素}

对象没有部分,但它可能有结构。结构由指向该对象的箭头定义。我们可以用箭头\emph{探测}对象。

在编程和逻辑学中,我们希望我们的初始对象没有结构。因此,我们假设它没有入箭头(除了从它自身循环回来的那个)。因此,\hask{Void}没有结构。

终止对象具有最简单的结构。从任何对象到它只有一个入箭头:从任何方向探测它只有一种方式。在这方面,终止对象表现得像一个不可分割的点。它唯一的属性是它存在,并且从任何其他对象到它的箭头证明了这一点。

因为终止对象如此简单,我们可以用它来探测其他更复杂的对象。

如果从终止对象到某个对象$a$有多个箭头,这意味着$a$有一些结构:有不止一种方式来看待它。由于终止对象表现得像一个点,我们可以将每个从它出发的箭头视为选择其目标的不同点或元素。

在范畴论中,我们说$ x$是$a$的\emph{全局元素},如果它是一个箭头

\[ 1 \xrightarrow x a \]

我们通常会简单地称它为元素(省略“全局”)。

在类型理论中,$ x \colon A$表示$x$是类型$A$。

在Haskell中,我们使用双冒号符号:
\begin{haskell}
x :: A
\end{haskell}
(Haskell使用大写字母表示具体类型,小写字母表示类型变量。)

我们说\hask{x}是类型\hask{A}的一个项,但从范畴论的角度,我们将其解释为一个箭头$x : 1 \to A$,即\hask{A}的全局元素。\footnote{Haskell类型系统区分\hask{x :: A}和\hask{x :: () -> A}。然而,它们在范畴语义中表示相同的东西。}

在逻辑学中,这样的$ x$被称为$ A$的证明,因为它对应于蕴涵$ \top \to A$(如果\textbf{True}为真,则\textbf{A}为真)。注意,$A$可能有许多不同的证明。

由于我们规定没有从任何其他对象到\hask{Void}的箭头,因此没有从终止对象到它的箭头。因此,\hask{Void}没有元素。这就是为什么我们认为\hask{Void}是空的。

终止对象只有一个元素,因为从它到自身有一个唯一的箭头,$ 1 \to 1$。这就是为什么我们有时称它为单例。

注意:在范畴论中,并不禁止初始对象有其他对象的入箭头。然而,在我们正在研究的笛卡尔闭范畴中,这是不允许的。

\section{箭头对象}

任何两个对象之间的箭头形成一个集合\footnote{严格来说,这仅在\emph{局部小}范畴中成立。}。这就是为什么一些集合论知识是学习范畴论的前提。

在编程中,我们谈论从\hask{a}到\hask{b}的函数的\emph{类型}。在Haskell中,我们写:
\begin{haskell}
f :: a -> b
\end{haskell}
意思是\hask{f}是“从\hask{a}到\hask{b}的函数”类型。这里,\hask{a->b}只是我们给这个类型的名称。

如果我们希望函数类型与其他类型一样被对待,我们需要一个\emph{对象}来表示从\hask{a}到\hask{b}的箭头集合。

要完全定义这个对象,我们必须描述它与其他对象的关系,特别是与\hask{a}和\hask{b}的关系。我们目前还没有工具来做到这一点,但我们会到达那里。

现在,让我们记住以下区别:一方面,我们有连接两个对象\hask{a}和\hask{b}的箭头。这些箭头形成一个集合。另一方面,我们有一个从\hask{a}到\hask{b}的\emph{箭头对象}。这个对象的“元素”被定义为从终止对象\hask{()}到我们称为\hask{a->b}的对象的箭头。

我们在编程中使用的符号往往模糊了这种区别。这就是为什么在范畴论中,我们将箭头对象称为\emph{指数},并将其写为$ b^a$(源对象在指数中)。因此,语句:
\begin{haskell}
f :: a -> b
\end{haskell}
等价于

\[ 1 \xrightarrow f b^a\]

在逻辑学中,一个箭头$ A \to B$是一个蕴涵:它陈述了“如果A则B”的事实。一个指数对象$ B^A$是对应的命题。它可能为真,也可能为假,我们不知道。你必须证明它。这样的证明是$ B^A$的一个元素。

给我一个$ B^A$的元素,我就知道$ B$可以从$ A$推导出来。

再次考虑“如果愿望是马,乞丐就会骑马”这句话——这次作为一个对象。它不是一个空对象,因为你可以指出它的证明——类似于:“有马的人会骑马。乞丐有愿望。既然愿望是马,乞丐就有马。因此乞丐会骑马。”但是,即使你有这个语句的证明,它对你也毫无用处,因为你永远无法证明它的前提:“愿望=马”。


\end{document}