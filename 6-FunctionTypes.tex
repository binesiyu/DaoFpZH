\documentclass[DaoFP]{subfiles}
\begin{document}
\setcounter{chapter}{5}

\chapter{函数类型}

在函数式编程中,还有另一种组合方式处于核心地位。它发生在你将一个函数作为参数传递给另一个函数时。外部函数可以将这个参数作为其自身机制的可插拔部分来使用。例如,它允许你实现一个通用的排序算法,该算法接受任意的比较函数。

如果我们将函数建模为对象之间的箭头,那么将函数作为参数意味着什么?

我们需要一种将函数对象化的方法,以便定义以“箭头对象”为源或目标的箭头。接受函数作为参数或返回函数的函数被称为\emph{高阶}函数。高阶函数是函数式编程的主力军。

\subsection{消去规则}

函数的一个定义特性是它可以应用于一个参数以产生结果。我们已经用组合的方式定义了函数应用:

\[
 \begin{tikzcd}
 1
 \arrow[d, "x"']
 \arrow[rd, "y"]
 \\
 a
 \arrow[r, "f"']
& b
 \end{tikzcd}
\]
这里$f$被表示为从$a$到$b$的箭头,但我们希望能够用箭头对象的元素替换$f$,或者如数学家所称的,指数对象$b^a$;或者如我们在编程中所称的,函数类型\hask{a->b}。

给定$b^a$的一个元素和$a$的一个元素,函数应用应该产生$b$的一个元素。换句话说,给定一对元素:
\begin{align*}
f &\colon 1 \to b^a \\
x &\colon 1 \to a
\end{align*}
它应该产生一个元素:
\[y \colon 1 \to b \]

请记住,这里$f$表示$b^a$的一个元素。之前,它是从$a$到$b$的箭头。

我们知道,一对元素$(f, x)$等价于积$b^a \times a$的一个元素。因此,我们可以将函数应用定义为单个箭头:
\[\varepsilon_{a b} \colon b^a \times a \to b\]
这样,应用的结果$y$由以下交换图定义:
\[
 \begin{tikzcd}
 1
 \arrow[d, "{(f, x)}"']
 \arrow[rd, "y"]
 \\
 b^a \times a
 \arrow[r, "\varepsilon_{a b}"']
& b
 \end{tikzcd}
\]
函数应用是函数类型的\emph{消去规则}。

当某人给你一个函数对象的元素时,你唯一能做的就是使用$\varepsilon$将其应用于参数类型的元素。

\subsection{引入规则}
为了完成函数对象的定义,我们还需要引入规则。

首先,假设有一种方法可以从某个其他对象$c$构造函数对象$b^a$。这意味着存在一个箭头
\[h \colon c \to b^a\]
我们知道可以使用$\varepsilon_{a b}$来消去$h$的结果,但我们必须首先将其乘以$a$。因此,让我们首先将$c$乘以$a$,然后使用函子性将其映射到$b^a \times a$。

函子性允许我们将一对箭头应用于积以得到另一个积。这里,这对箭头是$(h, id_a)$(我们希望将$c$变为$b^a$,但我们不感兴趣修改$a$)
\[ c \times a \xrightarrow{h \times id_a} b^a \times a \]
我们现在可以跟随这个箭头进行函数应用,以到达$b$
\[ c \times a \xrightarrow{h \times id_a} b^a \times a \xrightarrow{\varepsilon_{a b}} b\]
这个复合箭头定义了一个我们称为$f$的映射:
\[f \colon c \times a \to b\]
这是对应的图

\[
 \begin{tikzcd}
 c \times a
 \arrow[d, dashed, "h \times id_a"']
 \arrow[rd, "f"]
 \\
 b^a \times a
 \arrow[r, "\varepsilon"']
& b
 \end{tikzcd}
\]
这个交换图告诉我们,给定一个$h$,我们可以构造一个$f$;但我们也可以要求相反的情况:每个映射$f \colon c \times a \to b$,应该唯一地定义一个映射到指数的$h \colon c \to b^a$。

我们可以使用这个性质,即两组箭头之间的一一对应关系,来定义指数对象。这是函数对象$b^a$的\emph{引入规则}。

我们已经看到,积是使用其映射入性质定义的。另一方面,函数应用被定义为积的\emph{映射出}。

\subsection{柯里化}

有几种方式可以看待这个定义。一种方式是将其视为柯里化的一个例子。

到目前为止,我们只考虑了单参数函数。这并不是一个真正的限制,因为我们总是可以将一个双参数函数实现为从积出发的(单参数)函数。函数对象定义中的$f$就是这样一个函数:
\begin{haskell}
f :: (c, a) -> b
\end{haskell}
另一方面,\hask{h}是一个返回函数的函数:
\begin{haskell}
h :: c -> (a -> b)
\end{haskell}
柯里化是这两组箭头之间的同构。

这种同构可以在Haskell中通过一对(高阶)函数来表示。由于在Haskell中,柯里化适用于任何类型,这些函数使用类型变量编写——它们是\emph{多态}的:
\begin{haskell}
curry   :: ((c, a) -> b)   -> (c -> (a -> b))
\end{haskell}

\begin{haskell}
uncurry :: (c -> (a -> b)) -> ((c, a) -> b)
\end{haskell}
换句话说,函数对象定义中的$h$可以写成
\[ h = curry\, f \]

当然,这样写时,\hask{curry}和\hask{uncurry}的类型对应于函数对象而不是箭头。这种区别通常被忽略,因为指数的\emph{元素}与定义它们的\emph{箭头}之间存在一一对应关系。当我们用终端对象替换任意对象$c$时,这一点很容易看到。我们得到:

\[
 \begin{tikzcd}
 1 \times a
 \arrow[d, dashed, "h \times id_a"']
 \arrow[rd, "f"]
 \\
 b^a \times a
 \arrow[r, "\varepsilon_{a b}"']
& b
 \end{tikzcd}
\]
在这种情况下,$h$是对象$b^a$的一个元素,$f$是从$1 \times a$到$b$的箭头。但我们知道$1 \times a$与$a$同构,因此,实际上,$f$是从$a$到$b$的箭头。

因此,从现在开始,我们将把箭头\hask{->}称为箭头$\to$,而不做太多区分。这种现象的正确说法是,该范畴是自丰富的。

我们可以将$\varepsilon_{a b}$写为Haskell函数\hask{apply}:
\begin{haskell}
apply :: (a -> b, a) -> b
apply (f, x) = f x
\end{haskell}
但这只是一个语法技巧:函数应用是内置在语言中的:\hask{f x}意味着\hask{f}应用于\hask{x}。其他编程语言要求函数的参数用括号括起来,但在Haskell中则不然。

尽管将函数应用定义为一个单独的函数可能看起来是多余的,但Haskell库确实为此提供了一个中缀运算符\hask{$}:
\begin{haskell}
($) :: (a -> b) -> a -> b
f $ x = f x
\end{haskell}
不过,技巧在于,常规的函数应用是左结合的,例如,\hask{f x y}与\hask{(f x) y}相同;但美元符号是右结合的,因此
\begin{haskell}
f  $ g x
\end{haskell}
与\hask{f (g x)}相同。在第一个例子中,\hask{f}必须是一个(至少)双参数的函数;在第二个例子中,它可以是一个单参数的函数。

在Haskell中,柯里化无处不在。双参数函数几乎总是写成一个返回函数的函数。由于函数箭头\hask{->}是右结合的,因此不需要为这种类型加括号。例如,对构造函数的签名是:
\begin{haskell}
pair :: a -> b -> (a, b)
\end{haskell}
你可以将其视为一个双参数函数返回一个对,或者一个单参数函数返回一个单参数函数,\hask{b->(a, b)}。这样,部分应用这样的函数是可以的,结果是另一个函数。例如,我们可以定义:
\begin{haskell}
pairWithTen :: a -> (Int, a)
pairWithTen = pair 10 -- pair的部分应用
\end{haskell}



\subsection{与λ演算的关系}

看待函数对象定义的另一种方式是将$c$解释为$f$定义的环境的类型。在这种情况下,通常将环境称为$\Gamma$。箭头被解释为使用$\Gamma$中定义的变量的表达式。

考虑一个简单的例子,表达式:
\[a x^2 + b x + c\]
你可以将其视为由实数三元组$(a, b, c)$和变量$x$参数化,假设$x$是一个复数。三元组是积$\mathbb{R} \times \mathbb{R} \times \mathbb{R}$的一个元素。这个积是我们表达式的环境$\Gamma$。

变量$x$是$\mathbb{C}$的一个元素。表达式是从积$\Gamma \times \mathbb{C}$到结果类型(这里也是$\mathbb{C}$)的箭头
\[f \colon \Gamma \times \mathbb{C} \to \mathbb{C} \]
这是一个从积出发的映射,因此我们可以用它来构造函数对象$\mathbb{C}^{\mathbb{C}}$并定义映射$h \colon \Gamma \to \mathbb{C}^{\mathbb{C}}$
\[
 \begin{tikzcd}
 \Gamma \times \mathbb{C}
 \arrow[d, dashed, "h \times id_{\mathbb{C}}"']
 \arrow[rd, "f"]
 \\
 \mathbb{C}^{\mathbb{C}} \times \mathbb{C}
 \arrow[r, "\varepsilon"']
& \mathbb{C}
 \end{tikzcd}
\]
这个新的映射$h$可以被视为函数对象的构造函数。生成的函数对象表示所有从$\mathbb{C}$到$\mathbb{C}$的函数,这些函数可以访问环境$\Gamma$;即,三元组参数$(a, b, c)$。

对应于我们原始的表达式$a x^2 + b x + c$,在$\mathbb{C}^{\mathbb{C}}$中有一个特定的函数,我们将其写为:
\[  \lambda x . \,a x^2 + b x + c \]
或者,在Haskell中,用\index{反斜杠}反斜杠替换$\lambda$,
\begin{haskell}
\x -> a * x^2 + b * x + c
\end{haskell}

箭头$h \colon \Gamma \to \mathbb{C}^{\mathbb{C}}$由箭头$f$唯一确定。这个映射产生一个我们称为$\lambda x . f$的函数。

一般来说,函数对象的定义图变为:
\[
 \begin{tikzcd}
 \Gamma \times a
 \arrow[d, dashed, "{h \times id_a}"']
 \arrow[rd, "f"]
 \\
 b^a \times a
 \arrow[r, "\varepsilon"']
& b
 \end{tikzcd}
\]

为表达式$f$提供自由参数的环境$\Gamma$是表示参数类型的多个对象的积(在我们的例子中,它是$\mathbb{R} \times \mathbb{R} \times \mathbb{R}$)。

空环境由终端对象$1$表示,即积的单位。在这种情况下,$f$只是一个箭头$a \to b$,而$h$简单地从函数对象$b^a$中选择一个对应于$f$的元素。

重要的是要记住,一般来说,函数对象表示依赖于外部参数的函数。这样的函数被称为\index{闭包}\emph{闭包}。闭包是从其环境中捕获值的函数。

这是我们的例子在Haskell中的翻译。对应于$f$,我们有一个表达式:
\begin{haskell}
(a :+ 0) * x * x + (b :+ 0) * x + (c :+ 0)
\end{haskell}
如果我们使用\hask{Double}来近似$\mathbb{R}$,我们的环境是积\hask{(Double, Double, Double)}。类型\hask{Complex}由另一个类型参数化——这里我们再次使用\hask{Double}:
\begin{haskell}
type C = Complex Double
\end{haskell}
从\hask{Double}到\hask{C}的转换是通过将虚部设为零来完成的,如\hask{(a :+ 0)}。

对应的箭头$h$接受环境并生成一个类型为\hask{C -> C}的闭包:
\begin{haskell}
h :: (Double, Double, Double) -> (C -> C)
h (a, b, c) = \x -> (a :+ 0) * x * x + (b :+ 0) * x + (c :+ 0)
\end{haskell}

\subsection{肯定前件}

在逻辑中,函数对象对应于蕴涵。从终端对象到函数对象的箭头是该蕴涵的证明。函数应用$\varepsilon$对应于逻辑学家所称的\emph{肯定前件}:如果你有蕴涵$A \Rightarrow B$的证明和$A$的证明,那么这就构成了$B$的证明。

\section{和与积的再探}

当函数获得与其他类型元素相同的地位时,我们就有工具直接将图转换为代码。

\subsection{和类型}

让我们从和的定义开始。
\[
 \begin{tikzcd}
 a
 \arrow[dr,  bend left, "\text{Left}"']
 \arrow[ddr, bend right, "f"']
 && b
 \arrow[dl, bend right, "\text{Right}"]
 \arrow[ddl, bend left, "g"]
 \\
&a + b
\arrow[d, dashed, "h"]
\\
& c
 \end{tikzcd}
\]
我们说,箭头对$(f, g)$唯一地确定了从和出发的映射$h$。我们可以使用高阶函数简洁地写出它:
\begin{haskell}
h = mapOut (f, g)
\end{haskell}
其中:
\begin{haskell}
mapOut :: (a -> c, b -> c) -> (Either a b -> c)
mapOut (f, g) = \aorb -> case aorb of
                         Left  a -> f a
                         Right b -> g b
\end{haskell}
这个函数接受一对函数作为参数,并返回一个函数。

首先,我们对对\hask{(f, g)}进行模式匹配以提取\hask{f}和\hask{g}。然后我们使用lambda构造一个新函数。这个lambda接受一个类型为\hask{Either a b}的参数,我们称之为\hask{aorb},并对其进行案例分析。如果它是用\hask{Left}构造的,我们将\hask{f}应用于其内容,否则我们将\hask{g}应用于其内容。

请注意,我们返回的函数是一个闭包。它从其环境中捕获\hask{f}和\hask{g}。

我们实现的函数紧密遵循图,但它不是用通常的Haskell风格编写的。Haskell程序员更喜欢对多参数函数进行柯里化。此外,如果可能,他们更喜欢消除lambda。

这是Haskell标准库中相同函数的版本,它被称为(小写)\hask{either}:
\begin{haskell}
either :: (a -> c) -> (b -> c) -> Either a b -> c
either f _ (Left x)     =  f x
either _ g (Right y)    =  g y
\end{haskell}

双射的另一个方向,从$h$到对$(f, g)$,也遵循图的箭头。
\begin{haskell}
unEither :: (Either a b -> c) -> (a -> c, b -> c)
unEither h = (h . Left, h . Right)
\end{haskell}


\subsection{积类型}

积类型由其映射入性质对偶定义。
\[
 \begin{tikzcd}
 & c
\arrow[d, dashed, "h"]
 \arrow[ddl, bend right, "f"']
 \arrow[ddr, bend left, "g"]
\\
&a \times b
 \arrow[dl,  "\text{fst}"]
  \arrow[dr,   "\text{snd}"']
\\
a && b
 \end{tikzcd}
\]
这是该图的直接Haskell解读
\begin{haskell}
mapIn :: (c -> a, c -> b) -> (c -> (a, b))
mapIn (f, g) = \c -> (f c, g c)
\end{haskell}
这是用Haskell风格编写的风格化版本,作为中缀运算符\hask{&&&}
\begin{haskell}
(&&&) :: (c -> a) -> (c -> b) -> (c -> (a, b))
(f &&& g) c = (f c, g c)
\end{haskell}
双射的另一个方向由以下给出:
\begin{haskell}
fork :: (c -> (a, b)) -> (c -> a, c -> b)
fork h = (fst . h, snd . h)
\end{haskell}
它也紧密遵循图的解读。

\subsection{函子性的再探}

和与积都是函子性的,这意味着我们可以将函数应用于它们的内容。我们准备将这些图转换为代码。

这是和类型的函子性:

\[
 \begin{tikzcd}
 a
 \arrow[d, "f"]
 \arrow[dr,  bend left, "\text{Left}"']
  && b
 \arrow[d, "g"]
 \arrow[dl, bend right, "\text{Right}"]
 \\
 a'
 \arrow[rd, "\text{Left}"']
&a + b
\arrow[d, dashed, "h"]
& b'
\arrow[ld, "\text{Right}"]
\\
& a' + b'
 \end{tikzcd}
\]
阅读这个图,我们可以立即使用\hask{either}写出$h$:
\begin{haskell}
h f g = either (Left . f) (Right . g)
\end{haskell}
或者我们可以扩展它并称之为\hask{bimap}:
\begin{haskell}
bimap :: (a -> a') -> (b -> b') -> Either a b -> Either a' b'
bimap f g (Left  a) = Left  (f a)
bimap f g (Right b) = Right (g b)
\end{haskell}
类似地,对于积类型:
\[
 \begin{tikzcd}
 & a \times b
\arrow[d, dashed, "h"]
 \arrow[dl,  "\text{fst}"']
 \arrow[dr,   "\text{snd}"]
\\
a
\arrow[d, "f"']
&a' \times b'
 \arrow[dl,  "\text{fst}"]
  \arrow[dr,   "\text{snd}"']
& b
\arrow[d, "g"]
\\
a' && b'
 \end{tikzcd}
\]
$h$可以写为:
\begin{haskell}
h f g = (f . fst) &&& (g . snd)
\end{haskell}
或者可以扩展为
\begin{haskell}
bimap :: (a -> a') -> (b -> b') -> (a, b) -> (a', b')
bimap f g (a, b) = (f a, g b)
\end{haskell}
在这两种情况下,我们都将这个高阶函数称为\hask{bimap},因为在Haskell中,和与积都是一个更一般类\hask{Bifunctor}的实例。

\section{函数类型的函子性}

函数类型,或称指数类型,也具有函子性,但有一个转折。我们关注的是从$b^a$到$b'^{a'}$的映射,其中带撇的对象通过某些箭头(待确定)与不带撇的对象相关联。

指数类型由其映射入性质定义,因此如果我们寻找
\[k \colon b^a \to b'^{a'} \]
我们应该绘制以$k$作为映射到$b'^{a'}$的图表。我们通过将$b^a$替换为$c$并将带撇的对象替换为不带撇的对象,从原始定义中得到这个图表:

\[
 \begin{tikzcd}
 b^a \times a'
 \arrow[d, dashed, "k \times id_a"']
 \arrow[rd, "g"]
 \\
 b'^{a'} \times a'
 \arrow[r, "\varepsilon"']
& b'
 \end{tikzcd}
\]
问题是:我们能否找到一个箭头$g$来完成这个图表?
\[g \colon b^a \times a' \to b'\]
如果我们找到这样的$g$,它将唯一地定义我们的$k$。

思考这个问题的方法是考虑如何实现$g$。它以乘积$b^a \times a'$作为其参数。将其视为一对:一个从$a$到$b$的函数对象元素和一个$a'$的元素。我们唯一能对函数对象做的事情是将其应用于某物。但$b^a$需要一个类型为$a$的参数,而我们手头只有$a'$。除非有人给我们一个箭头$a' \to a$,否则我们无法做任何事情。这个箭头应用于$a'$将生成$b^a$的参数。然而,应用的结果是类型$b$,而$g$应该产生一个$b'$。同样,我们需要一个箭头$b \to b'$来完成我们的任务。

这听起来可能很复杂,但归根结底,我们需要在带撇和不带撇的对象之间有两个箭头。转折在于第一个箭头从$a'$到$a$,这感觉与通常的函子性考虑相反。为了将$b^a$映射到$b'^{a'}$,我们需要一对箭头:
\begin{align*}
f &\colon a' \to a \\
g &\colon b \to b' 
\end{align*}

这在Haskell中更容易解释。我们的目标是实现一个函数\hask{a' -> b'},给定一个函数\hask{h :: a -> b}。

这个新函数接受一个类型为\hask{a'}的参数,因此在我们将其传递给\hask{h}之前,我们需要将\hask{a'}转换为\hask{a}。这就是为什么我们需要一个函数\hask{f :: a' -> a}。

由于\hask{h}产生一个\hask{b},而我们希望返回一个\hask{b'},我们需要另一个函数\hask{g :: b -> b'}。所有这些都很好地融入一个高阶函数:
\begin{haskell}
dimap :: (a' -> a) -> (b -> b') -> (a -> b) -> (a' -> b')
dimap f g h = g . h . f
\end{haskell}
类似于\hask{bimap}是类型类\hask{Bifunctor}的接口,\hask{dimap}是类型类\hask{Profunctor}的成员。

\section{双笛卡尔闭范畴}

在范畴论中,如果一个范畴对任意两个对象都定义了积和指数,并且具有终对象,则该范畴称为\emph{笛卡尔闭范畴}。其核心思想是,hom集并不是该范畴的外来概念:范畴在形成hom集的操作下是“闭”的。

如果该范畴还具有和(余积)和始对象,则称为\emph{双笛卡尔闭范畴}。

这是建模编程语言所需的最小结构。

使用这些操作构造的数据类型称为\emph{代数数据类型}。我们拥有类型的加法、乘法和指数运算(但没有减法或除法);并且满足我们在高中代数中熟悉的所有定律。这些定律在同构的意义下成立。还有一个我们尚未讨论的代数定律。

\subsection{分配律}

数的乘法对加法具有分配性。在双笛卡尔闭范畴中,我们是否应该期望同样的性质?
\[b \times a + c \times a \cong (b + c) \times a\]

从左到右的映射很容易构造,因为它同时是一个从和出发的映射和一个到积的映射。我们可以通过逐步将其分解为更简单的映射来构造它。在Haskell中,这意味着实现一个函数
\begin{haskell}
dist :: Either (b, a) (c, a) -> (Either b c, a)
\end{haskell}
从左侧的和出发的映射由一对箭头给出:
\begin{align*}
f &\colon b\times a \to (b + c) \times a \\
g &\colon c\times a \to (b + c) \times a 
\end{align*}
\[
 \begin{tikzcd}
 b \times a
 \arrow[dr,  bend left, "\text{Left}"]
 \arrow[ddr, bend right, "f"']
 && c \times a
 \arrow[dl, bend right, "\text{Right}"']
 \arrow[ddl, bend left, "g"]
 \\
& b \times a + c \times a
\arrow[d, dashed, "\text{dist}"]
\\
& (b + c) \times a
 \end{tikzcd}
\]

我们在Haskell中将其写为:
\begin{haskell}
dist = either f g
  where
    f   :: (b, a) -> (Either b c, a)
    g   :: (c, a) -> (Either b c, a)
\end{haskell}
\index{\hask{where}}\hask{where}子句用于引入子函数的定义。

现在我们需要实现$f$和$g$。它们都是到积的映射,因此每个都等价于一对箭头。例如,第一个由以下箭头对给出:
\begin{align*}
f' &\colon b \times a \to (b + c) \\
f'' &\colon b \times a \to  a
\end{align*}
\[
 \begin{tikzcd}
 & b \times a
\arrow[d, dashed, "f"]
 \arrow[ddl, bend right, "f'"']
 \arrow[ddr, bend left, "f''"]
\\
& (b + c) \times a
 \arrow[dl,  "\text{fst}"]
  \arrow[dr,   "\text{snd}"']
\\
b + c && a
 \end{tikzcd}
\]

在Haskell中:
\begin{haskell}
    f = f' &&& f''
    f'  :: (b, a) -> Either b c
    f'' :: (b, a) -> a
\end{haskell}
第一个箭头可以通过投影第一个分量$b$,然后使用$\text{Left}$来构造和。第二个箭头只是投影$\text{snd}$:
\begin{align*}
 f' &= \text{Left} \circ \text{fst} \\
 f'' &= \text{snd}
\end{align*}
类似地,我们将$g$分解为$g'$和$g''$:
\[
 \begin{tikzcd}
 & c \times a
\arrow[d, dashed, "g"]
 \arrow[ddl, bend right, "g'"']
 \arrow[ddr, bend left, "g''"]
\\
& b + c \times a
 \arrow[dl,  "\text{fst}"]
  \arrow[dr,   "\text{snd}"']
\\
(b + c) && a
 \end{tikzcd}
\]

将这些组合在一起,我们得到:
\begin{haskell}
dist = either f g
  where
    f   = f' &&& f''
    f'  = Left . fst
    f'' = snd
    g   = g' &&& g''
    g'  = Right . fst
    g'' = snd
\end{haskell}
这些是辅助函数的类型签名:
\begin{haskell}
    f   :: (b, a) -> (Either b c, a)
    g   :: (c, a) -> (Either b c, a)
    f'  :: (b, a) -> Either b c
    f'' :: (b, a) -> a
    g'  :: (c, a) -> Either b c
    g'' :: (c, a) -> a
\end{haskell}
它们也可以内联以产生这种简洁的形式:
\begin{haskell}
dist = either ((Left . fst) &&& snd) ((Right . fst) &&& snd)
\end{haskell}

这种编程风格称为\emph{无点}风格,因为它省略了参数(点)。出于可读性考虑,Haskell程序员更喜欢更显式的风格。上述函数通常会实现为:
\begin{haskell}
dist (Left  (b, a)) = (Left  b, a)
dist (Right (c, a)) = (Right c, a)
\end{haskell}

注意,我们只使用了和与积的定义。同构的另一个方向需要使用指数,因此它仅在双笛卡尔\emph{闭}范畴中有效。这在直接的Haskell实现中并不明显:
\begin{haskell}
undist :: (Either b c, a) -> Either (b, a) (c, a)
undist (Left b, a)  = Left (b, a)
undist (Right c, a) = Right (c, a)
\end{haskell}
但这是因为在Haskell中柯里化是隐式的。

这是该函数的无点版本:
\begin{haskell}
undist = uncurry (either (curry Left) (curry Right))
\end{haskell}
这可能不是最易读的实现,但它强调了我们需要指数的事实:我们使用\hask{curry}和\hask{uncurry}来实现映射。

我们将在稍后回到这个恒等式,那时我们将拥有更强大的工具:伴随。

\begin{exercise}
证明:
\[ 2 \times a \cong a + a \]
其中$2$是布尔类型。首先用图解法证明,然后实现两个Haskell函数来见证这个同构。
\end{exercise}

\end{document}