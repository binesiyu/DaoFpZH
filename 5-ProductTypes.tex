\documentclass[DaoFP]{subfiles}
\begin{document}
\setcounter{chapter}{4}

\chapter{积类型}

我们可以使用和类型(sum types)来枚举给定类型的可能值,但这种编码方式可能比较浪费。我们仅仅为了编码0到9的数字,就需要十个构造函数。
\begin{haskell}
data Digit = Zero | One | Two | Three | ... | Nine
\end{haskell}
但是,如果我们将两个数字组合成一个数据结构,即一个两位的十进制数,我们就能够编码一百个数字。或者,正如老子所说,只需四个数字,你就能编码一万个数字。

以这种方式组合两种类型的数据类型称为积(product),或\index{cartesian product}\emph{笛卡尔}积。其定义性质是消去规则:从$a \times b$出发有两个箭头;一个称为``$\text{fst}$''的箭头指向$a$,另一个称为``$\text{snd}$''的箭头指向$b$。它们被称为(笛卡尔)\index{cartesian projections}\emph{投影}。它们让我们可以从积$a \times b$中提取$a$和$b$。

\[
 \begin{tikzcd}
& a \times b
 \arrow[dl,  "\text{fst}"]
 \arrow[dr,   "\text{snd}"']
\\
a && b
 \end{tikzcd}
\]

假设有人给了你一个积的元素,即从终对象$1$到$a \times b$的箭头$h$。你可以很容易地通过组合提取出一对元素:$a$的元素由
\[x = \text{fst} \circ h \]
给出,$b$的元素由
\[y = \text{snd} \circ h \]
给出。

\[
 \begin{tikzcd}
 & 1
\arrow[d, dashed, "h"]
 \arrow[ddl, bend right, "x"']
 \arrow[ddr, bend left, "y"]
\\
&a \times b
 \arrow[dl,  "\text{fst}"]
 \arrow[dr,   "\text{snd}"']
\\
a && b
 \end{tikzcd}
\]

事实上,给定从任意对象$c$到$a \times b$的箭头,我们可以通过组合定义一对箭头$f \colon c \to a$和$g \colon c \to b$

\[
 \begin{tikzcd}
 & c
\arrow[d, dashed, "h"]
 \arrow[ddl, bend right, "f"']
 \arrow[ddr, bend left, "g"]
\\
&a \times b
 \arrow[dl,  "\text{fst}"]
  \arrow[dr,   "\text{snd}"']
\\
a && b
 \end{tikzcd}
\]

正如我们之前对和类型所做的那样,我们可以将这个想法反过来,用这个图来\emph{定义}积类型:我们施加一个条件,即一对函数$f$和$g$与从$c$到$a \times b$的\index{mapping in}\emph{映射入}一一对应。这是积的\emph{引入}规则。

特别是,从终对象的映射在Haskell中用于定义积类型。给定两个元素,\hask{a :: A}和\hask{b :: B},我们构造积

\begin{haskell}
(a, b) :: (A, B)
\end{haskell}
积的内置语法就是这样:一对括号和中间的一个逗号。它既可以用于定义两种类型的积\hask{(A, B)},也可以用于定义数据构造函数\hask{(a, b)},它接受两个元素并将它们配对在一起。

我们永远不应忘记编程的目的:将复杂问题分解为一系列更简单的问题。我们在积的定义中再次看到了这一点。每当我们必须构造一个\emph{映射入}积时,我们将其分解为构造一对函数的两个较小任务,每个函数映射到积的一个组件中。这就像说,要实现一个返回一对值的函数,只需实现两个函数,每个函数返回该对中的一个元素。

\subsection{逻辑}

在逻辑中,积类型对应于逻辑合取。为了证明$A \times B$($A$与$B$),你需要提供\emph{两者}$A$和$B$的证明。这些是目标为$A$和$B$的箭头。消去规则说,如果你有$A \times B$的证明,那么你自动得到$A$的证明(通过$\text{fst}$)和$B$的证明(通过$\text{snd}$)。

\subsection{元组和记录}

正如老子所说,一万个对象的积只是一个有一万个投影的对象。

我们可以在Haskell中使用元组符号形成任意积。例如,三种类型的积写为\hask{(A, B, C)}。这种类型的项可以从三个元素构造:\hask{(a, b, c)}。

在数学家称之为“符号滥用”的情况下,零种类型的积写为\hask{()},一个空元组,它恰好与终对象或单位类型相同。这是因为积的行为非常类似于数字的乘法,终对象扮演着1的角色。

在Haskell中,我们使用模式匹配语法,而不是为所有元组定义单独的投影。例如,要从三元组中提取第三个组件,我们会写

\begin{haskell}
thrd :: (a, b, c) -> c
thrd (_, _, c) = c
\end{haskell}
我们使用通配符来忽略我们想要忽略的组件。

老子说:“名者,万物之始也。”在编程中,如果不给特定元组的组件命名,就很难跟踪它们的含义。记录语法允许我们为投影命名。这是用记录风格编写的积的定义:
\begin{haskell}
data Product a b = Pair { fst :: a, snd :: b }
\end{haskell}
\hask{Pair}是数据构造函数,\hask{fst}和\hask{snd}是投影。

这是它如何用于声明和初始化特定对的示例:
\begin{haskell}
ic :: Product Int Char
ic = Pair 10 'A'
\end{haskell}

\section{笛卡尔范畴}

在Haskell中,我们可以定义任意两种类型的积。一个所有积都存在且终对象存在的范畴称为\emph{笛卡尔}范畴。

\subsection{元组算术}

积满足的恒等式可以使用映射入性质推导出来。例如,要证明$a \times b \cong b \times a$,考虑以下两个图:

\[
 \begin{tikzcd}
 &x
 \arrow[d, dashed, "h"]
 \arrow[ddl, bend right, "f"']
 \arrow[ddr, bend left, "g"]
 \\
 & a \times b
  \arrow[dl,  "\text{fst}"]
 \arrow[dr,   "\text{snd}"']
 \\
a && b
 \end{tikzcd}
 \qquad
 \begin{tikzcd}
 &x
 \arrow[d, dashed, "h'"]
 \arrow[ddl, bend right, "g"']
 \arrow[ddr, bend left, "f"]
 \\
 & b \times a
  \arrow[dl,  "\text{fst}"]
 \arrow[dr,   "\text{snd}"']
\\
b && a
  \end{tikzcd}
\]

它们表明,对于任何对象$x$,到$a \times b$的箭头与到$b \times a$的箭头一一对应。这是因为这些箭头中的每一个都由相同的对$f$和$g$决定。

你可以检查自然性条件是否满足,因为当你使用$k \colon x' \to x$改变视角时,所有从$x$出发的箭头都通过预组合$(- \circ k)$进行了移位。

在Haskell中,这种同构可以实现为一个函数,它是它自己的逆函数:
\begin{haskell}
swap :: (a, b) -> (b, a)
swap x = (snd x, fst x)
\end{haskell}
这是使用模式匹配编写的相同函数:
\begin{haskell}
swap (x, y) = (y, x)
\end{haskell}

重要的是要记住,积的对称性只是“在同构意义下”。这并不意味着交换对的顺序不会改变程序的行为。对称性意味着交换对的信息内容相同,但访问它需要修改。

终对象是积的单位,$1 \times a \cong a$。见证$1 \times a$和$a$之间同构的箭头称为\emph{左单位元}:
\[ \lambda \colon 1 \times a \to a \]
它可以实现为$\lambda = \text{snd}$。它的逆$\lambda^{-1}$定义为下图中的唯一箭头:
\[
 \begin{tikzcd}
 &a
 \arrow[d, dashed, "\lambda^{-1}"]
 \arrow[ddl, bend right, "!"']
 \arrow[ddr, bend left, "\text{id}"]
 \\
 & 1 \times a
  \arrow[dl,  "\text{fst}"]
 \arrow[dr,   "\text{snd}"']
 \\
1 && a
 \end{tikzcd}
 \]
从$a$到$1$的箭头称为\index{!}$!$(发音为\emph{bang})。这确实表明
\[\text{snd} \circ \lambda^{-1} = \text{id} \] 
我们仍然需要证明$\lambda^{-1}$是$\text{snd}$的左逆。考虑下图:
\[
 \begin{tikzcd}
 &1 \times a
 \arrow[d, dashed, "h"]
 \arrow[ddl, bend right, "!"']
 \arrow[ddr, bend left, "\text{snd}"]
 \\
 & 1 \times a
  \arrow[dl,  "\text{fst}"]
 \arrow[dr,   "\text{snd}"']
 \\
1 && a
 \end{tikzcd}
 \]
 对于$h = id$,它显然交换。对于$h = \lambda^{-1}  \circ  \text{snd}$,它也交换,因为我们有:
 \[  \text{snd} \circ \lambda^{-1}  \circ  \text{snd} = \text{snd} \]
 由于$h$应该是唯一的,我们得出结论:
\[ \lambda^{-1}  \circ  \text{snd} = id \]
这种使用普遍构造的推理是相当标准的。

以下是一些用Haskell编写的其他同构(没有逆的证明)。这是结合律:
\begin{haskell}
assoc :: ((a, b), c) -> (a, (b, c))
assoc ((a, b), c) = (a, (b, c))
\end{haskell}
这是右单位元
\begin{haskell}
runit :: (a, ()) -> a
runit (a, _) = a
\end{haskell}

这两个函数对应于\index{associator}\emph{结合子}
\[ \alpha \colon (a \times b) \times c \to a \times (b \times c) \]
和\index{unitor}\emph{右单位元}:
\[ \rho \colon a \times 1 \to a \]

\begin{exercise}
证明左单位元证明中的双射是自然的。提示,使用箭头$g \colon a \to b$改变焦点。
\end{exercise}

\begin{exercise}
构造一个箭头
\[ h \colon b + a \times b \to (1 + a) \times b \]
这个箭头是唯一的吗?

提示:它是一个映射入积,所以它由一对箭头给出。这些箭头又映射出和,所以每个箭头由一对箭头给出。

提示:映射$b \to 1 + a$由$(\text{Left} \, \circ \, !)$给出
\end{exercise}

\begin{exercise}
重做上一个练习,这次将$h$视为\emph{映射出}和。
\end{exercise}

\begin{exercise}
实现一个Haskell函数\hask{maybeAB :: Either b (a, b) -> (Maybe a, b)}。这个函数是否由其类型签名唯一定义,还是有一些自由度?
\end{exercise}

\subsection{函子性}

假设我们有将$a$和$b$映射到某个$a'$和$b'$的箭头:
\begin{align*}
f &\colon a \to a' \\
g &\colon b \to b'
\end{align*}
这些箭头与投影$\text{fst}$和$\text{snd}$的组合可以用于定义积之间的映射$h$:

\[
 \begin{tikzcd}
 & a \times b
\arrow[d, dashed, "h"]
 \arrow[dl,  "\text{fst}"']
 \arrow[dr,   "\text{snd}"]
\\
a
\arrow[d, "f"']
&a' \times b'
 \arrow[dl,  "\text{fst}"]
  \arrow[dr,   "\text{snd}"']
& b
\arrow[d, "g"]
\\
a' && b'
 \end{tikzcd}
\]
这个图的简写符号是:
\[ a \times b \xrightarrow{f \times g} a' \times b' \]

积的这种性质称为\emph{函子性}。你可以想象它允许你转换积\emph{内部}的两个对象以获得新的积。我们也说函子性让我们\index{lifting}\emph{提升}一对箭头以操作积。

\section{对偶性}

当一个孩子看到箭头时,他知道哪一端指向源(source),哪一端指向目标(target)
\[a \to b \]
但这可能只是一种先入为主的观念。如果我们称$b$为源(source)而$a$为目标(target),宇宙会变得非常不同吗?

我们仍然可以将这个箭头与另一个箭头组合
\[b \to c\]
其"目标"$b$与$a \to b$的"源"相同,结果仍然是一个箭头
\[a \to c\]
只是现在我们会说它从$c$指向$a$。

在这个对偶的宇宙中,我们称为"初始"(initial)的对象将被称为"终结"(terminal),因为它是来自所有对象的唯一箭头的"目标"。相反,终结对象将被称为初始。

现在考虑我们用来定义和对象(sum object)的图表:
\[
 \begin{tikzcd}
 a
 \arrow[dr,  bend left, "\text{Left}"']
 \arrow[ddr, bend right, "f"']
 && b
 \arrow[dl, bend right, "\text{Right}"]
 \arrow[ddl, bend left, "g"]
 \\
&a + b
\arrow[d, dashed, "h"]
\\
& c
 \end{tikzcd}
\]
在新的解释中,箭头$h$将从任意对象$c$指向我们称为$a + b$的对象。这个箭头由一对"源"为$c$的箭头$(f, g)$唯一确定。如果我们将$\text{Left}$重命名为$\text{fst}$,将$\text{Right}$重命名为$\text{snd}$,我们将得到积对象(product)的定义图表。

积对象是箭头反向的和对象。

相反,和对象是箭头反向的积对象。

\medskip

范畴论中的每个构造都有其对偶。

\medskip

如果箭头的方向只是一个解释问题,那么在编程中,和类型(sum types)与积类型(product types)为何如此不同?这种差异可以追溯到我们一开始做出的一个假设:初始对象没有传入的箭头(除了恒等箭头)。这与终结对象有许多传出箭头形成对比,我们使用这些箭头来定义(全局)元素。事实上,我们假设每个感兴趣的对象都有元素,而没有元素的对象与\hask{Void}同构。

当我们讨论函数类型时,我们将看到一个更深刻的差异。

\section{幺半范畴}

我们已经看到积满足以下简单规则:
\begin{align*}
1 \times a &\cong a
\\
a \times b &\cong b \times a
\\
(a \times b) \times c &\cong a \times (b \times c)
\end{align*}
并且是函子性的。

定义了具有这些性质的运算的范畴称为\emph{对称幺半范畴}\footnote{严格来说,两个对象的积是定义在同构意义下的,而幺半范畴中的积必须精确定义。但我们可以通过选择一个积来得到一个幺半范畴}。我们之前在讨论和与初始对象时已经见过类似的结构。

一个范畴可以同时具有多个幺半结构。当不想为幺半结构命名时,可以用张量符号代替加号或积符号,用字母$I$代替中性元素。于是,对称幺半范畴的规则可以写成:
\begin{align*}
I \otimes a &\cong a
\\
a \otimes b &\cong b \otimes a
\\
(a \otimes b) \otimes c &\cong a \otimes (b \otimes c)
\end{align*}

这些同构通常写成可逆箭头的族,称为\index{结合子}结合子和\index{单位子}单位子。如果幺半范畴不是对称的,则存在独立的左单位子和右单位子。
\begin{align*}
\alpha &\colon (a \otimes b) \otimes c \to a \otimes (b \otimes c)
\\
 \lambda &\colon I \otimes a \to a
 \\
 \rho &\colon a \otimes I \to a
\end{align*}
对称性由以下箭头体现:
\[ \gamma \colon a \otimes b \to b \otimes a \]
函子性允许我们将一对箭头\emph{提升}:
\begin{align*} 
f &\colon a \to a' \\
g &\colon b \to b'
\end{align*}
以作用于张量积上:
\[ a \otimes b \xrightarrow{f \otimes g} a' \otimes b' \]

如果我们将态射视为动作,它们的张量积对应于并行执行两个动作。这与态射的串行组合形成对比,后者暗示了它们的时间顺序。

你可以将张量积视为积与和的最低公分母。它仍然有一个引入规则,需要两个对象$a$和$b$;但它没有消去规则。一旦创建,张量积就“忘记”了它是如何创建的。与笛卡尔积不同,它没有投影。

一些有趣的张量积例子甚至不是对称的。

\subsection{幺半群}

幺半群是非常简单的结构,配备了一个二元运算和一个单位元。带有加法和零的自然数形成一个幺半群。带有乘法和一的自然数也是如此。

直观上,幺半群允许你将两个事物组合成另一个事物。还有一个特殊的事物,与任何其他事物组合都会返回相同的事物。这就是单位元。并且组合必须是结合的。

不假设的是组合是对称的,或者存在逆元。

定义幺半群的规则让人想起范畴的规则。不同之处在于,在幺半群中,任何两个事物都是可组合的,而在范畴中通常不是这样:只有当两个箭头的目标是一个箭头的源时,才能组合它们。除非范畴只包含一个对象,在这种情况下所有箭头都是可组合的。

只有一个对象的范畴称为幺半群。组合运算是箭头的组合,单位元是恒等箭头。

这是一个完全有效的定义。然而,在实践中,我们通常对嵌入在更大范畴中的幺半群感兴趣。特别是在编程中,我们希望能够在类型和函数的范畴内定义幺半群。

然而,在范畴中,我们更喜欢批量定义运算,而不是查看单个元素。因此,我们从一个对象$m$开始。二元运算是两个参数的函数。由于积的元素是成对的元素,我们可以将二元运算描述为从积$m \times m$到$m$的箭头:
\[ \mu \colon m \times m \to m \]
单位元素可以定义为从终对象$1$出发的箭头:
\[ \eta \colon 1 \to m \]

我们可以通过定义一个\index{类型类}类型类来直接将这个描述翻译成Haskell,该类配备了两个方法,传统上称为\hask{mappend}和\hask{mempty}:
\begin{haskell}
class Monoid m where
  mappend :: (m, m) -> m
  mempty  :: () -> m
\end{haskell}

两个箭头$\mu$和$\eta$必须满足幺半群定律,但同样,我们必须批量表述它们,而不依赖于任何元素。

为了表述左单位律,我们首先创建积$1 \times m$。然后使用$\eta$“在$m$中选择单位元”,或者用箭头的话说,将$1$转换为$m$。由于我们在积$1 \times m$上操作,我们必须提升对$\langle \eta, id_m \rangle$,这确保我们“不触碰”$m$。最后,我们使用$\mu$执行“乘法”。

我们希望结果与$m$的原始元素相同,但不提及元素。因此,我们只需使用左单位子$\lambda$从$1 \times m$到$m$,而不“搅动事物”。
\[
 \begin{tikzcd}
 1 \times m
 \arrow[rr, "\eta \times id_m"]
 \arrow[rrd, "\lambda"']
& & m \times m
 \arrow[d, "\mu"]
 \\
 && m
  \end{tikzcd}
\]
这是右单位的类似定律:
\[
 \begin{tikzcd}
 m \times m
 \arrow[d, "\mu"]
 && m \times 1
 \arrow[ll, "id_m \times \eta"']
 \arrow[lld, "\rho"]
 \\
 m
 \end{tikzcd}
\]
为了表述结合律,我们必须从三重积开始并批量操作。这里,$\alpha$是结合子,它重新排列积而不“搅动事物”。
\[
 \begin{tikzcd}
 (m \times m) \times m 
 \arrow[rr, "\alpha"]
 \arrow[d, "\mu \times id"]
 &&
 m \times (m \times m)
 \arrow[d, "id \times \mu"]
 \\
 m \times m 
 \arrow[dr, "\mu"]
& & m \times m
 \arrow[dl, "\mu"']
 \\
&  m
 \end{tikzcd}
\]

注意,我们不必对与对象$m$和$1$一起使用的范畴积做很多假设。特别是我们从未使用过投影。这表明上述定义在任意幺半范畴中的张量积上同样适用。它甚至不必是对称的。我们只需假设:存在一个单位对象,积是函子性的,并且它满足单位和结合律的同构。

因此,如果我们将$\times$替换为$\otimes$,将$1$替换为$I$,我们就得到了在任意幺半范畴中幺半群的定义。

\index{幺半群}\emph{幺半群}在幺半范畴中是一个对象$m$,配备两个态射:
\[ \mu \colon m \otimes m \to m \]
\[ \eta \colon I \to m \]
满足单位和结合律:
\[
 \begin{tikzcd}
 1 \otimes m
 \arrow[rr, "\eta \otimes id_m"]
 \arrow[rrd, "\lambda"']
& & m \otimes m
 \arrow[d, "\mu"']
&& m \otimes 1
 \arrow[ll, "id_m \otimes \eta"']
 \arrow[lld, "\rho"]
 \\
 && m
  \end{tikzcd}
\]

\[
 \begin{tikzcd}
 (m \otimes m) \otimes m 
 \arrow[rr, "\alpha"]
 \arrow[d, "\mu \otimes id_m"]
 &&
 m \otimes (m \otimes m)
 \arrow[d, "id_m \otimes \mu"]
 \\
 m \otimes m 
 \arrow[dr, "\mu"]
& & m \otimes m
 \arrow[dl, "\mu"']
 \\
&  m
 \end{tikzcd}
\]
我们使用了$\otimes$的函子性来提升箭头对,如$\eta \otimes id_m$,$\mu \otimes id_m$等。

\end{document}