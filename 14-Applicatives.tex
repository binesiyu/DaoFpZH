\documentclass[DaoFP]{subfiles}
\begin{document}
\setcounter{chapter}{13}

\chapter{应用函子(Applicative Functors)}

老子会说:"如果你并行执行两件事,所需时间将减半。"

\section{并行组合}

当我们将范畴论的语言翻译成编程语言时,我们将箭头视为计算。但计算需要时间,这引入了一个新的时间维度。这反映在我们使用的语言中。例如,两个箭头的组合结果 $g \circ f$,通常被读作 $g$ \emph{在} $f$ \emph{之后}。$g$ 的执行必须 \emph{跟随} $f$ 的执行,因为我们需要 $f$ 的结果才能调用 $g$。我们称之为 \index{sequential composition}\emph{顺序组合}。它反映了计算之间的依赖关系。

然而,有些计算至少原则上可以并行执行,因为它们之间没有依赖关系。这只有在并行计算的结果可以组合时才有意义,这意味着我们需要在幺半群范畴中工作。因此,并行组合的基本构建块是两个箭头的张量积,$f \colon x \to a$ 和 $g \colon y \to b$:
\[ x \otimes y \xrightarrow{f \otimes g} a \otimes b \]
在笛卡尔范畴中,我们通常使用笛卡尔积作为张量。在编程中,我们只需运行两个函数,然后组合它们的结果。

\subsection{幺半群函子}

当我们尝试组合\emph{有副作用}的计算结果时,问题就出现了。为此,我们需要一种方法来组合一对结果 \hask{f a} 和 \hask{f b} 以及它们的副作用,以生成单个结果 \hask{f (a, b)}。当然,每种类型的副作用需要不同的处理方式。

让我们以部分性(partiality)副作用为例。我们可以使用以下方式组合两个这样的副作用:
\begin{haskell}
combine :: Maybe a -> Maybe b -> Maybe (a, b)
combine (Just a) (Just b) = Just (a, b)
combine _ _ = Nothing
\end{haskell}
只有当两个计算都成功时,结果才是成功的。

一个计算还应该能够产生一个"忽略我"的结果。这将是一个返回单位值(即相对于笛卡尔积的单位)和"忽略我"副作用的计算。注意,返回 \hask{Nothing} 是不行的,因为它会使另一个计算的结果无效。正确的实现是:
\begin{haskell}
ignoreMe :: Maybe ()
ignoreMe = Just ()
\end{haskell}
当与任何其他部分计算 \hask{combine}'d 时,它会被忽略(符合左/右单位律)。

一般来说,如果类型构造器 \hask{f} 是 \hask{Monoidal} 类的实例,它就支持并行组合:
\begin{haskell}
class Monoidal f where
  unit  :: f ()
  (>*<) :: f a -> f b -> f (a, b)
\end{haskell}
特别是,\hask{Maybe} 函子是 \hask{Monoidal} 的:
\begin{haskell}
instance Monoidal Maybe where
  unit  = Just ()
  Just a >*< Just b = Just (a, b)
  _ >*< _ = Nothing
\end{haskell}
幺半群函子必须与笛卡尔范畴的幺半群结构兼容,因此它们应满足明显的单位和结合律。

\subsection{应用函子}

尽管 \hask{Monoidal} 类为副作用的并行组合提供了足够的支持,但它并不十分实用。

首先,我们需要能够对组合结果进行操作,因此我们需要 \hask{f} 的 \hask{Functor} 实例。这让我们可以将类型为 \hask{(a, b)->c } 的函数应用于有副作用的对 \hask{f(a, b)},而不影响副作用。

一旦我们知道如何并行运行两个计算,我们就可以利用结合律来组合任意数量的计算。例如:
\begin{haskell}
run3 :: (Functor f, Monoidal f) => 
  (x -> f a) -> (y -> f b) -> (z -> f c) -> 
  (x, y, z) -> f (a, b, c)
run3 f1 f2 f3 (x, y, z) =
  let fab  = f1 x >*< f2 y
      fabc = fab >*< f3 z
  in fmap reassoc fabc
\end{haskell}
其中我们重新关联结果的三元组:
\begin{haskell}
reassoc :: ((a, b), c) -> (a, b, c)
reassoc ((a, b), c) = (a, b, c)
\end{haskell}
\index{\hask{let}}\hask{let} 子句用于引入局部绑定。在这里,局部变量 \hask{fab} 和 \hask{fabc} 被初始化为相应的幺半群积。\hask{let}/\hask{in} 构造是一个表达式,其值由 \hask{in} 子句的内容给出。

但随着计算数量的增加,事情变得越来越笨拙。

幸运的是,有一种更符合人体工程学的方法。在大多数应用中,并行计算的结果被输入到一个函数中,该函数在生成最终结果之前收集它们。我们需要的是提升这种多参数函数的方法——这是 \hask{Functor} 执行的单参数函数提升的推广。

例如,要组合两个并行计算的结果,我们可以使用函数:
\begin{haskell}
liftA2 :: (a -> b -> c) -> f a -> f b -> f c
\end{haskell}
不幸的是,我们需要无限多个这样的函数,以适应所有可能的参数数量。诀窍是用一个利用柯里化的函数来替换它们。

一个有两个参数的函数是一个返回函数的单参数函数。因此,假设 \hask{f} 是一个函子,我们可以先 \hask{fmap}'ping:
\begin{haskell}
a -> (b -> c)
\end{haskell}
在第一个参数 \hask{(f a)} 上得到:
\begin{haskell}
f (b -> c)
\end{haskell}
现在我们需要将 \hask{f (b -> c)} 应用于第二个参数 \hask{(f b)}。如果函子是 \hask{Monoidal},我们可以使用操作符 \hask{>*< } 将两者组合,得到类型为:
\begin{haskell}
f (b -> c, b)
\end{haskell}
然后我们可以 \hask{fmap} 函数应用 \hask{apply} 到它上面:
\begin{haskell}
fmap apply (ff >*< fa)
\end{haskell}
其中:
\begin{haskell}
apply :: (a -> b, a) -> b
apply (f, a) = f a
\end{haskell}

或者,我们可以定义中缀操作符 \hask{<*>},它让我们在组合副作用的同时将函数应用于参数:
\begin{haskell}
(<*>) :: f (a -> b) -> f a -> f b
fs <*> as = fmap apply (fs >*< as)
\end{haskell}
这个操作符有时被称为 \index{\hask{<*>}}\index{splat}``splat''。

相反,我们可以用 splat 来实现幺半群操作符:
\begin{haskell}
fa >*< fb = fmap (,) fa <*> fb
\end{haskell}
其中我们使用了对构造器 \hask{(,)} 作为双参数函数。

使用 \hask{<*>} 而不是 \hask{>*<} 的优势在于,它让我们可以通过重复剥离柯里化层次来应用任意数量参数的函数。

例如,我们可以将三个参数的函数 \hask{g :: a -> b -> c -> d } 应用于三个值 \hask{fa :: f a }, \hask{fb :: f b }, 和 \hask{fc :: f c },使用:
\begin{haskell}
fmap g fa <*> fb <*> fc
\end{haskell}
我们甚至可以使用 \hask{fmap} 的中缀版本 \index{\hask{<$>}}\hask{<$>} 使其看起来更像函数应用:
\begin{haskell}
g <$> fa <*> fb <*> fc
\end{haskell}
其中:
\begin{haskell}
(<$>) :: Functor f => (a -> b) -> f a -> f b
(<$>) = fmap
\end{haskell}
这两个操作符都向左绑定,因此我们可以将此符号视为函数应用的直接推广:
\begin{haskell}
g a b c
\end{haskell}
除了它还累积了三个计算的副作用。

为了完善这个图景,我们还需要定义应用一个有副作用的零参数函数的含义。它只是意味着在一个单一值上附加一个"忽略我"的副作用。我们可以使用幺半群的 \hask{unit} 来实现这一点:
\begin{haskell}
pure :: a -> f a
pure a = fmap (const a) unit
\end{haskell}
相反,\hask{unit} 可以用 \hask{pure} 来实现:
\begin{haskell}
unit = pure ()
\end{haskell}

因此,在笛卡尔闭范畴中,我们可以使用等价的但更方便的 \hask{Applicative} 来代替 \hask{Monoidal}:
\begin{haskell}
class Functor f => Applicative f where
    pure  :: a -> f a
    (<*>) :: f (a -> b) -> f a -> f b
\end{haskell}

\section{应用函子实例}

最常见的应用函子同时也是单子(最著名的反例是\hask{ZipList})。通常选择使用\hask{Applicative}还是\hask{Monad}语法是基于便利性,尽管在某些情况下存在性能差异——尤其是在涉及并行执行时。

\subsection{部分性}
一个\hask{Maybe}函数只有在两者都是\hask{Just}时才能应用于\hask{Maybe}参数:
\begin{haskell}
instance Applicative Maybe where
  pure = Just
  mf <*> ma = 
    case (mf, ma) of
      (Just f, Just a) -> Just (f a)
      _ -> Nothing
\end{haskell}

我们可以通过定义\hask{Either}函子的版本来添加关于失败原因的信息:
\begin{haskell}
data Validation e a = Failure e  | Success a
\end{haskell}
\hask{Validation}的\hask{Applicative}实例使用\hask{mappend}累积错误:
\begin{haskell}
instance Monoid e => Applicative (Validation e) where
  pure = Success
  Failure e1 <*> Failure e2 = Failure (mappend e1 e2)
  Failure e  <*> Success _ = Failure e
  Success _  <*> Failure e  = Failure  e
  Success f  <*> Success x  = Success (f x)
\end{haskell}

\subsection{日志记录}

\begin{haskell}
newtype Writer w a = Writer { runWriter :: (a, w) }
  deriving Functor
\end{haskell}

为了使\hask{Writer}函子支持组合,我们必须能够组合记录的值,并且有一个“忽略我”的元素。这意味着日志必须是一个幺半群:
\begin{haskell}
instance Monoid w => Applicative (Writer w) where
  pure a = Writer (a, mempty)
  wf <*> wa = let (f, w)  = runWriter wf
                  (a, w') = runWriter wa
              in Writer (f a, mappend w w')
\end{haskell}

\subsection{环境}
\begin{haskell}
newtype Reader e a = Reader { runReader :: e -> a }
  deriving Functor
\end{haskell}

由于环境是不可变的,我们并行传递给两个计算。无效果的\hask{pure}忽略环境。
\begin{haskell}
instance Applicative (Reader e) where
  pure a = Reader (const a)
  rf <*> ra = Reader (\e -> (runReader rf e) (runReader ra e))
\end{haskell}

\subsection{状态}
\begin{haskell}
newtype State s a = State { runState :: s -> (a, s) }
  deriving Functor
\end{haskell}

\hask{State}应用函子展示了并行和顺序组合。两个动作是并行创建的,但它们是顺序执行的。第二个动作使用第一个动作修改后的状态:
\begin{haskell}
instance Applicative (State s) where
  pure a = State (\s -> (a, s))
  sf <*> sa = State (\s ->
    let (f, s')  = runState sf s
        (a, s'') = runState sa s' 
    in (f a, s''))
\end{haskell}

\subsection{非确定性}
列表函子有两个独立的\hask{Applicative}实例。它们对应于两种不同的列表组合方式。第一种是逐个元素压缩两个列表;第二种产生所有可能的组合。为了对同一数据类型有两个实例,我们必须将其中一个封装在\hask{newtype}中:
\begin{haskell}
newtype ZipList a = ZipList { unZip :: [a] }
  deriving Functor
\end{haskell}
splat操作符将每个函数应用于其对应的参数。它在较短列表的末尾停止。有趣的是,如果我们希望\hask{pure}符合单位律,它必须产生一个无限列表:
\begin{haskell}
repeat :: a -> [a]
repeat a = a : repeat a
\end{haskell}
这样,当与任何有限列表压缩时,它不会截断结果。
\begin{haskell}
instance Applicative ZipList where
  pure a = ZipList (repeat a)
  zf <*> za = ZipList (zipWith ($) (unZip zf) (unZip za))
\end{haskell}
两个(可能无限的)列表通过应用函数应用操作符\hask{$}组合。
\begin{haskell}
zipWith :: (a -> b -> c) -> [a] -> [b] -> [c]
zipWith f [] _ = []
zipWith f _ [] = []
zipWith f (a : as) (b : bs) = f a b : zipWith f as bs 
\end{haskell}

列表的第二个\hask{Applicative}实例将所有函数应用于所有参数。为了实现splat操作符,我们使用Haskell的\index{列表推导}列表推导语法:
\begin{haskell}
instance Applicative [] where
  pure a = [a]
  fs <*> as = [ f a | f <- fs, a <- as ]
\end{haskell}
\hask{[f a | f <- fs, a <- as]}的含义是:创建一个\hask{(f a)}的列表,其中\hask{f}取自列表\hask{fs},\hask{a}取自列表\hask{as}。

\subsection{延续}
\begin{haskell}
newtype Cont r a = Cont { runCont :: (a -> r) -> r }
  deriving Functor
\end{haskell}

让我们为延续实现\hask{<*>}操作符。作用于一对延续
\begin{haskell}
kf :: Cont r (a -> b)
ka :: Cont r a
\end{haskell}
它应该产生一个延续:
\begin{haskell}
Cont r b
\end{haskell}
后者接受一个处理程序\hask{k :: b -> r },并应该用函数应用\hask{(f a)}的结果调用它。为了进行最后一次调用,我们需要提取函数和参数。为了提取\hask{a},我们必须运行延续\hask{ka}并传递\hask{a}的消费者:
\begin{haskell}
runCont ka (\a -> k (f a))))
\end{haskell}
为了提取\hask{f},我们必须运行延续\hask{kf}并传递\hask{f}的消费者:
\begin{haskell}
runCont kf (\f -> runCont ka (\a -> k (f a))))
\end{haskell}
总的来说,我们得到:
\begin{haskell}
instance Applicative (Cont r) where
  pure a = Cont (\k -> k a)
  kf <*> ka = Cont (\k -> 
        runCont kf (\f -> 
        runCont ka (\a -> k (f a))))
\end{haskell}

\subsection{输入/输出}

输入/输出操作通常作为单子组合,但也可以使用应用语法。注意,尽管应用函子是并行组合的,但副作用是序列化的。这很重要,例如,如果你希望在获取用户输入之前打印提示:
\begin{haskell}
prompt :: String -> IO String
prompt str = putStrLn str *> getLine
\end{haskell}
这里我们组合了\hask{putStrLn}(将其参数打印到终端)和\hask{getLine}(等待用户输入)。半splat操作符忽略第一个参数产生的值(在这种情况下是单位\hask{()}),但保留副作用(这里,打印字符串):
\begin{haskell}
(*>) :: Applicative f => f a -> f b -> f b
u *> v = (\ _ x -> x) <$> u <*> v
\end{haskell}
我们现在可以应用一个双参数函数:
\begin{haskell}
greeting :: String -> String -> String
greeting s1 s2 = "Hi " ++ s1 ++ " " ++ s2 ++ "!"
\end{haskell}
到一对\hask{IO}参数:
\begin{haskell}
getNamesAndGreet :: IO String
getNamesAndGreet = 
    greeting <$> prompt "First name: " <*> prompt "Last name: "
\end{haskell}

\subsection{解析器}

解析是一种高度可分解的活动。有许多领域特定语言可以使用应用解析器(而不是更强大的单子解析器)进行解析。

解析器可以描述为接受字符(或标记)字符串的函数。解析可能失败,因此结果是\hask{Maybe}。成功的解析器返回解析值以及未消耗的输入字符串部分:
\begin{haskell}
newtype Parser a = 
  Parser { parse :: String -> Maybe (a, String) }
\end{haskell}
例如,这是一个检测数字的简单解析器:
\begin{haskell}
digit :: Parser Char
digit = Parser (\s -> case s of
    (c:cs) | isDigit c -> Just (c, cs)
    _                  -> Nothing)
\end{haskell}
解析器的\hask{Applicative}实例将部分性与状态结合起来:
\begin{haskell}
instance Applicative Parser where
  pure a = Parser (\s -> Just (a, s))
  pf <*> pa = Parser (\s ->
    case parse pf s of
      Nothing      -> Nothing
      Just (f, s') -> case parse pa s' of
          Nothing       -> Nothing
          Just (a, s'') -> Just (f a, s''))
\end{haskell}

大多数语法包含替代项,例如,标识符或数字;if语句或循环等。这可以通过\hask{Applicative}的以下\index{\hask{Alternative}}子类捕获:
\begin{haskell}
class Applicative f => Alternative f where
  empty :: f a
  (<|>) :: f a -> f a -> f a
\end{haskell}
这表明,无论类型\hask{a}是什么,类型\hask{(f a)}都是一个幺半群。

\hask{Parser}的\hask{Alternative}实例尝试第一个解析器,如果失败,则尝试第二个。幺半群单位是一个总是失败的解析器。
\begin{haskell}
instance Alternative Parser where
  empty = Parser (\s -> Nothing)
  pa <|> pb = Parser (\s ->
    case parse pa s of
      Just as -> Just as
      Nothing -> parse pb s)
 \end{haskell}
 如果我们使用\hask{Maybe}的\hask{Alternative}实例,这可以简化:
\begin{haskell}
  pa <|> pb = Parser (\s -> parse pa s <|> parse pb s)
 \end{haskell}
 拥有\hask{Applicative}超类允许我们链式替代项。我们可以使用\hask{some}来解析一个或多个项目:
\begin{haskell}
some :: f a -> f [a] 
some pa = (:) <$> pa <*> many pa
\end{haskell}
和\hask{many}来解析零个或多个项目:
\begin{haskell}
many :: f a -> f [a]
many pa = some pa <|> pure []
\end{haskell}

\begin{exercise}
实现\hask{Maybe}的\hask{Alternative}实例。提示:如果第一个参数是失败,尝试第二个。
\end{exercise}

\subsection{并发和并行}

启动线程是一个I/O操作,因此任何并发或并行执行本质上都是有副作用的。计算是并行还是串行执行取决于它们的组合方式。并行性是通过应用组合实现的。

\hask{Concurrently}函子是一个用于并行操作的\hask{Applicative}示例。它接受一个\hask{IO}操作并在单独的线程中开始执行它。然后使用应用组合来组合结果——这里,通过添加两个整数:
\begin{haskell}
f :: Concurrently Int
f = (+) <$> Concurrently (fileChars "1-Types.hs") 
        <*> Concurrently (fileChars "2-Composition.hs")
    where fileChars path = length <$> readFile path
\end{haskell}

\subsection{Do Notation}
在范畴论中,我们以图的方式组合箭头。我们可以将它们串联或并联,并且不需要为中间对象的元素命名。这样的图可以直接翻译为编程,产生用于串联组合的无点符号和用于并行组合的应用符号。然而,命名中间结果通常很方便。这可以使用\hask{do}符号完成。例如,并发示例可以重写为:
\begin{haskell}
{-# language ApplicativeDo #-}
f :: Concurrently Int
f = do
      m <- Concurrently (fileChars "1-Types.hs")
      n <- Concurrently (fileChars "2-Composition.hs")
      pure (m + n)
  where fileChars path = length <$> readFile path
\end{haskell}

我们将在关于单子的章节中更多地讨论do符号。现在,重要的是要知道,当你使用语言编译指示\hask{ApplicativeDo}时,编译器默认会尝试在do块的翻译中使用应用组合。如果不可能,它将回退到单子组合。

\subsection{应用函子的组合}

应用函子的一个很好的特性是它们的函子组合再次是一个应用函子:
\begin{haskell}
instance (Applicative f, Applicative g) => 
  Applicative (Compose g f) where
    pure :: a -> Compose g f a
    pure x = Compose (pure (pure x))
    (<*>) :: Compose g f (a -> b) -> Compose g f a -> Compose g f b
    Compose gff <*> Compose gfa = Compose (fmap (<*>) gff <*> gfa)
\end{haskell}
我们稍后会看到,单子并非如此:两个单子的组合通常不是单子。

\section{范畴论中的幺半函子}

我们已经看到了几个幺半范畴(monoidal category)的例子。这类范畴配备了某种二元运算,例如笛卡尔积、和、复合(在自函子范畴中)等。它们还有一个特殊对象,作为该二元运算的单位元。单位元和结合律要么严格满足(在严格幺半范畴中),要么在同构意义下满足。

每当我们有多个某种结构的实例时,我们可能会问自己一个问题:是否存在一个由这些事物构成的完整范畴?在这种情况下:幺半范畴能否形成它们自己的范畴?为此,我们必须定义幺半范畴之间的箭头。

一个从幺半范畴 $(\mathcal{C}, \otimes, i)$ 到另一个幺半范畴 $(\mathcal{D}, \oplus, j)$ 的\emph{幺半函子}(monoidal functor)$F$ 将张量积映射到张量积,单位元映射到单位元——所有这些都在同构意义下成立:
\begin{align*}
F a \oplus F b &\cong F (a \otimes b) \\
j &\cong F i 
\end{align*}
这里,左边是目标范畴中的张量积和单位元,右边是源范畴中的对应物。

如果所讨论的两个幺半范畴不是严格的,即单位元和结合律仅在同构意义下满足,那么还需要额外的相干性条件,以确保单位元映射到单位元,结合子映射到结合子。

以幺半函子为箭头的幺半范畴的范畴称为 $\mathbf{MonCat}$。实际上,它是一个2-范畴,因为可以定义幺半函子之间的结构保持自然变换。

\subsection{松弛幺半函子}

幺半范畴的一个优点是可以让我们定义幺半群(monoid)。你可以很容易地确信,幺半函子将幺半群映射到幺半群。事实证明,你不需要幺半函子的全部功能来实现这一点。让我们考虑一个函子将幺半群映射到幺半群所需的最小条件。

让我们从幺半范畴 $(\mathcal{C}, \otimes, i)$ 中的一个幺半群 $(m, \mu, \eta)$ 开始。考虑一个将 $m$ 映射到 $F m$ 的函子 $F$。我们希望 $F m$ 成为目标幺半范畴 $(\mathcal{D}, \oplus, j)$ 中的一个幺半群。为此,我们需要找到两个映射:
\begin{align*}
\eta' &\colon j \to F m \\
 \mu' &\colon F m \oplus F m \to F m 
\end{align*}
满足幺半群定律。

由于 $m$ 是一个幺半群,我们确实可以利用原始映射的提升:
\begin{align*}
 F \eta &\colon F i \to F m \\
 F \mu &\colon F (m \otimes m) \to F m
\end{align*}

为了实现 $\eta'$ 和 $\mu'$,我们缺少的是两个额外的箭头:
\begin{align*}
j &\to F i\\
 F m \oplus F m &\to F (m \otimes m)
 \end{align*}
一个幺半函子会提供这样的箭头。然而,对于我们试图实现的目标,我们不需要这些箭头是可逆的。

一个\emph{松弛幺半函子}(lax monoidal functor)是一个配备了一个态射 $\phi_i$ 和一个自然变换 $\phi_{ab}$ 的函子:
\begin{align*}
\phi_i &\colon j \to F i \\
\phi_{a b} &\colon F a \oplus F b \to F (a \otimes b)
\end{align*}
满足适当的单位性和结合性条件。

这样的函子将幺半群 $(m, \mu, \eta)$ 映射到幺半群 $(F m, \mu', \eta')$,其中:
\begin{align*}
\eta' &= F \eta \circ \phi_i \\
\mu' &= F \mu \circ \phi_{a b}
\end{align*}

在 Haskell 中,我们将 \hask{Monoidal} 函子视为一个松弛幺半自函子的例子,它保留了笛卡尔积。

\subsection{函子强度}

函子与幺半结构交互的另一种方式,在我们编程时隐藏得很明显。我们理所当然地认为函数可以访问环境。这样的函数称为闭包(closure)。

例如,这里有一个函数从环境中捕获变量 \hask{a} 并将其与其参数配对:
\begin{haskell}
\x -> (a, x)
\end{haskell}
这个定义单独来看没有意义,但当环境中包含变量 \hask{a} 时,它就有意义了,例如:
\begin{haskell}
pairWith :: Int -> (String -> (Int, String))
pairWith a = \x -> (a, x)
\end{haskell}
调用 \hask{pairWith 5} 返回的函数从其环境中“捕获”了 5。

现在考虑以下修改,它返回一个包含闭包的单例列表:
\begin{haskell}
pairWith' :: Int -> [String -> (Int, String)]
pairWith' a = [\x -> (a, x)]
\end{haskell}
作为程序员,如果这不起作用,你会非常惊讶。但我们在这里做的事情非常不平凡:我们正在将环境“偷运”到列表函子下。根据我们的 lambda 演算模型,闭包是从环境和函数参数的乘积到结果的态射。这里的 lambda,实际上是 \hask{(Int, String)} 的函数,定义在列表函子内部,但它捕获了在列表\emph{外部}定义的 \hask{a}。

让我们将环境偷运到函子下的属性称为\index{强度,函子}\emph{函子强度}(functorial strength)或\emph{张量强度}(tensorial strength),可以在 Haskell 中实现为:
\begin{haskell}
strength :: Functor f => (e, f a) -> f (e, a)
strength (e, as) = fmap (e, ) as
\end{haskell}
符号 \hask{(e, )} 称为\index{元组部分}\emph{元组部分}(tuple section),等价于对构造函数的部分应用:\hask{(,) e}。

在范畴论中,自函子 $F$ 的强度被定义为一个将张量积偷运到函子中的自然变换:
\[ \sigma \colon a \otimes F(b) \to F (a \otimes b) \]
还有一些额外的条件,确保它与所讨论的幺半范畴的单位元和结合子良好地配合。

我们能够为任意函子实现 \hask{strength} 的事实意味着,在 Haskell 中,每个函子都是强的。这就是为什么我们不必担心从函子内部访问环境的原因。

然而,在范畴论中,并非幺半范畴中的每个自函子都是强的。目前,我们的魔法咒语是我们正在使用的范畴是自丰富的(self-enriched),并且在 Haskell 中定义的每个自函子都是丰富的。当我们讨论丰富范畴时,我们会回到这一点。在 Haskell 中,强度归结为我们总是可以 \hask{fmap} 一个部分应用的元组构造函数 \hask{(a, )}。

\subsection{闭函子}

如果你眯着眼睛看 splat 操作符的定义:
\begin{haskell}
(<*>) :: f (a -> b) -> (f a -> f b)
\end{haskell}
你可能会将其视为将函数对象映射到函数对象。

如果你考虑两个闭范畴之间的函子,这一点会变得更加清晰。你可以从源范畴中的函数对象 $b^a$ 开始,并对其应用函子 $F$:
\[ F (b^a) \]
或者,你可以映射两个对象 $a$ 和 $b$,并在目标范畴中构造它们之间的函数对象:
\[ (F b)^{F a} \]
如果我们要求这两种方式同构,我们就得到了严格\emph{闭函子}(closed functor)的定义。但是,就像幺半函子的情况一样,我们更感兴趣的是松弛版本,它配备了一个单向自然变换:
\[ F (b^a) \to (F b)^{F a} \]
如果 $F$ 是一个自函子,这直接转化为 splat 操作符的定义。

松弛闭函子的完整定义包括幺半单位元的映射和一些相干性条件。综上所述,应用函子(applicative functor)就是一个松弛闭函子。

\end{document}