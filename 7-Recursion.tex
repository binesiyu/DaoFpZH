\documentclass[DaoFP]{subfiles}
\begin{document}
\setcounter{chapter}{6}

\chapter{递归}

``道生一。

一生二。

二生三。

三生万物。''


\vspace{10pt}

当你站在两面镜子之间时,你会看到自己的倒影,倒影的倒影,倒影的倒影的倒影,依此类推。每个倒影都是根据前一个倒影定义的,但它们共同产生了无限。

递归是一种分解模式,它将单个任务分解为多个步骤,步骤的数量可能是无限的。

递归基于对怀疑的悬置。你面临的任务可能需要任意多个步骤。你暂时假设你知道如何解决它。然后你问自己一个问题:“如果我已经解决了除最后一步之外的所有问题,我将如何迈出最后一步?”

\section{自然数}

自然数对象$N$并不包含数字。对象没有内部结构。结构由箭头定义。

我们可以使用从终端对象出发的箭头来定义一个特殊元素。按照惯例,我们将这个箭头称为$Z$,表示“零”。
\[ Z \colon 1 \to N \]
但我们必须能够定义无限多个箭头,以说明对于每个自然数,都有一个比它大一的数。

我们可以通过以下方式形式化这一陈述:假设我们知道如何创建一个自然数$n \colon 1 \to N$。我们如何迈出下一步,指向下一个数——它的后继?

这个下一步可以简单地将$n$与一个从$N$回到$N$的箭头进行后复合。这个箭头不应该是恒等箭头,因为我们希望一个数的后继与该数不同。但一个这样的箭头,我们称之为$S$,表示“后继”,就足够了。

对应于$n$的后继的元素由以下复合给出:
\[ 1 \xrightarrow{n} N \xrightarrow{S} N\]
(有时我们在单个图中多次绘制同一个对象,以便拉直循环箭头。)

特别地,我们可以将$One$定义为$Z$的后继:
\[
 \begin{tikzcd}
 1
 \arrow[rr, bend left, "One"]
 \arrow[r, "Z"]
 &N
  \arrow[r, "S"]
&N
  \end{tikzcd}
\]
将$Two$定义为$Z$的后继的后继:
\[
 \begin{tikzcd}
 1
 \arrow[rrr, bend left, "Two"]
 \arrow[r, "Z"]
 &N
  \arrow[r, "S"]
&N
  \arrow[r, "S"]
 &N
 \end{tikzcd}
\]
依此类推。

\subsection{引入规则}

两个箭头$Z$和$S$作为自然数对象$N$的引入规则。其中的一个是递归的:$S$使用$N$作为其源和目标。

\[
 \begin{tikzcd}
 1
 \arrow[r, "Z"]
 &N
 \arrow[out=135, in=45, loop, "S"]
 \end{tikzcd}
\]
这两个引入规则直接翻译为Haskell:

\begin{haskell}
data Nat where
  Z :: Nat
  S :: Nat -> Nat
\end{haskell}
它们可以用来定义任意的自然数;例如:

\begin{haskell}
zero, one, two :: Nat
zero = Z
one  = S zero
two  = S one
\end{haskell}

这种自然数类型的定义在实际中并不十分有用。然而,它通常用于定义类型级别的自然数,其中每个数字都是其自己的类型。

你可能会在\index{皮亚诺数}\emph{皮亚诺算术}的名称下遇到这种构造。

\subsection{消除规则}

引入规则是递归的这一事实在定义消除规则时稍微复杂了一些。我们将遵循前面章节的模式,首先假设我们有一个从$N$出发的映射:
\[ h \colon N \to a \]
然后看看我们能从中推断出什么。

之前,我们能够将这样的$h$分解为更简单的映射(对于和与积的映射对;对于指数的积的映射)。

$N$的引入规则看起来与和的引入规则相似(它要么是$Z$,要么是后继),因此我们期望$h$可以拆分为两个箭头。确实,我们可以通过复合$h \circ Z$轻松得到第一个箭头。这是一个选择$a$元素的箭头。我们称之为$\mathit{init}$:
\[\mathit{init} \colon 1 \to a \]
但没有明显的方法找到第二个箭头。

为了看到这一点,让我们扩展$N$的定义:
\[
 \begin{tikzcd}
 1
 \arrow[r, "Z"]
 &N
  \arrow[r, "S"]
&N
  \arrow[r, "S"]
&N
& ...
  \end{tikzcd}
\]
并将$h$和$\text{init}$插入其中:
\[
 \begin{tikzcd}
 1
 \arrow[r, "Z"]
 \arrow[rd, "\mathit{init}"']
 &N
  \arrow[r, "S"]
\arrow[d, dashed, "h"]
&N
  \arrow[r, "S"]
\arrow[d, dashed, "h"]
&N
\arrow[d, dashed, "h"]
& ...
\\
& a
& a
& a
  \end{tikzcd}
\]

直觉上,从$N$到$a$的箭头表示$a$的元素序列$a_n$。第零个元素由
\[a_0=\mathit{init}\]
给出。下一个元素是
\[a_1 = h \circ S \circ Z \]
接着是
\[a_2 = h \circ S \circ S \circ Z \]
依此类推。

因此,我们用一个箭头$h$替换了无限多个箭头$a_n$。诚然,新的箭头更简单,因为它们表示$a$的元素,但它们的数量是无限的。

问题在于,无论如何看待它,从$N$出发的任意映射都包含无限的信息。

我们必须大幅简化问题。由于我们使用单个箭头$S$生成所有自然数,我们可以尝试使用单个箭头$a \to a$生成所有元素$a_n$。我们将这个箭头称为$\mathit{step}$:
\[
 \begin{tikzcd}
 1
 \arrow[r, "Z"]
 \arrow[rd, "\mathit{init}"']
 &N
  \arrow[r, "S"]
\arrow[d, dashed, "h"]
&N
\arrow[d, dashed, "h"]
\\
& a
\arrow[r, "\mathit{step}"]
& a
  \end{tikzcd}
\]
由这样的对$\mathit{init}$和$\mathit{step}$生成的从$N$出发的映射称为\emph{递归的}。并非所有从$N$出发的映射都是递归的。事实上,递归映射非常少;但递归映射足以定义自然数对象。

我们使用上述图作为消除规则。我们规定,每个从$N$出发的递归映射$h$与一对$\mathit{init}$和$\mathit{step}$一一对应。

这意味着\emph{求值规则}(为给定的$h$提取$(\mathit{init}, \mathit{step})$)不能为任意箭头$h \colon N \to a$制定,只能为那些先前使用$(\mathit{init}, \mathit{step})$递归定义的箭头制定。

箭头$\mathit{init}$总是可以通过复合$h \circ Z$恢复。箭头$\mathit{step}$是以下方程的解:
\[ \mathit{step} \circ h = h \circ S \]
如果$h$是使用某个$\mathit{init}$和$\mathit{step}$定义的,那么这个方程显然有解。

重要的是,我们要求这个解是\emph{唯一的}。

直觉上,对$\mathit{init}$和$\mathit{step}$生成元素序列$a_0$,$a_1$,$a_2$,...如果两个箭头$h$和$h'$由相同的对$(\mathit{init}, \mathit{step})$给出,这意味着它们生成的序列是相同的。

因此,如果$h$与$h'$不同,这意味着$N$包含的不仅仅是元素序列$Z$,$S Z$,$S(S Z)$,...例如,如果我们将$-1$添加到$N$中(即让$Z$成为某人的后继),我们可能会有$h$和$h'$在$-1$处不同,但由相同的$\mathit{init}$和$\mathit{step}$生成。唯一性意味着在$Z$和$S$生成的数字之前、之后或之间没有自然数。

我们在这里讨论的消除规则对应于\emph{原始递归}。我们将在关于依赖类型的章节中看到这个规则的更高级版本,对应于归纳原理。

\subsection{在编程中}

消除规则可以在Haskell中实现为递归函数:

\begin{haskell}
rec :: a -> (a -> a) -> (Nat -> a)
rec init step = \n ->
  case n of
    Z     -> init
    (S m) -> step (rec init step m)
\end{haskell}

这个单一的函数,称为\emph{递归器},足以实现所有自然数的递归函数。例如,这是我们如何实现加法:

\begin{haskell}
plus :: Nat -> Nat -> Nat
plus n = rec init step
  where
    init = n
    step = S
\end{haskell}
这个函数将\hask{n}作为参数,并生成一个函数(闭包),该函数接受另一个数字并将其与\hask{n}相加。

在实践中,程序员更喜欢直接实现递归——这种方法相当于内联递归器\hask{rec}。以下实现可能更容易理解:
\begin{haskell}
plus n m = case m of
  Z -> n
  (S k) -> S (plus k n)
\end{haskell}
可以理解为:如果\hask{m}为零,则结果为\hask{n}。否则,如果\hask{m}是某个\hask{k}的后继,则结果是\hask{k + n}的后继。这与说\hask{init = n}和\hask{step = S}完全相同。

在命令式语言中,递归通常被迭代取代。从概念上讲,迭代似乎更容易理解,因为它对应于顺序分解。序列中的步骤通常遵循某种自然顺序。这与递归分解形成对比,在递归分解中,我们假设我们已经完成了到第$n$步的所有工作,并将该结果与下一个连续步骤结合起来。

另一方面,递归在处理递归定义的数据结构(如列表或树)时更为自然。

这两种方法是等价的,编译器通常将递归函数转换为循环,这称为\emph{尾递归优化}。

\begin{exercise}
使用递归器实现一个将\hask{Nat}转换为\hask{Int}的函数。
\end{exercise}

\begin{exercise}
将加法的柯里化版本实现为从$N$到函数对象$N^N$的映射。提示:在递归器中使用这些类型:
\begin{haskell}
init :: Nat -> Nat
step :: (Nat -> Nat) -> (Nat -> Nat)
\end{haskell}

\end{exercise}

\section{列表}

事物列表要么为空,要么是一个事物后跟一个事物列表。

这个递归定义转化为类型$L_a$($a$的列表)的两个引入规则:
\begin{align*}
 \text{Nil} &\colon 1 \to L_a \\
 \text{Cons} &\colon a \times L_a \to L_a 
\end{align*}
$\text{Nil}$元素描述一个空列表,$\text{Cons}$从头部和尾部构造一个列表。

以下图描述了投影和列表构造函数之间的关系。投影提取使用$\text{Cons}$构造的列表的头部和尾部。
\[
 \begin{tikzcd}
 & a \times L_a
 \arrow[ld, "fst"]
 \arrow[rd,  "snd"']
 \arrow[rd, bend left=60, red, "\text{Cons}"]
 && 1
 \arrow[ld, red, "\text{Nil}"]
 \\
 a
&&L_a
  \end{tikzcd}
\]

这个描述可以立即翻译为Haskell:
\begin{haskell}
data List a where
  Nil  :: List a
  Cons :: (a, List a) -> List a
\end{haskell}



\subsection{消除规则}

假设我们有一个从$a$的列表到某个任意类型$c$的映射:
\[h \colon L_a \to c\]
这是我们如何将其插入列表定义的方式:
\[
 \begin{tikzcd}
 1
 \arrow[r, "\text{Nil}"]
 \arrow[rd, "\mathit{init}"']
 &L_a
\arrow[d, dashed, "h"]
&a \times L_a
  \arrow[l, "\text{Cons}"']
\arrow[d, dashed, "id_a \times h"]
\\
& c
& a \times c
\arrow[l, "\mathit{step}"]
  \end{tikzcd}
\]
我们使用积的函子性将$(id_a, h)$应用于积$a \times L_a$。

类似于自然数对象,我们可以尝试定义两个箭头,$\mathit{init} = h \circ \text{Nil}$和$\mathit{step}$。箭头$\mathit{step}$是以下方程的解:
\[ \mathit{step} \circ (id_a \times h) = h \circ \text{Cons} \]
同样,并非每个$h$都可以简化为这样的一对箭头。

然而,给定$\mathit{init}$和$\mathit{step}$,我们可以定义一个$h$。这样的函数称为\emph{折叠},或列表的范畴同态。

这是Haskell中的列表递归器:
\begin{haskell}
recList :: c -> ((a, c) -> c) -> (List a -> c)
recList init step = \as ->
  case as of 
    Nil          -> init
    Cons (a, as) -> step (a, recList init step as)
\end{haskell}
给定\hask{init}和\hask{step},它生成一个从列表出发的映射。

列表是如此基本的数据类型,以至于Haskell为其内置了语法。类型\hask{(List a)}写为\hask{[a]}。\hask{Nil}构造函数是一对空的方括号,\hask{[]},\hask{Cons}构造函数是中缀冒号\hask{(:)}。

我们可以对这些构造函数进行模式匹配。从列表出发的通用映射具有以下形式:
\begin{haskell}
h :: [a] -> c
h []      = -- 空列表的情况
h (a : as) = -- 非空列表的头部和尾部的情况
\end{haskell}

对应于列表递归器\hask{recList},以下是标准库中函数\hask{foldr}(右折叠)的类型签名:
\begin{haskell}
foldr :: (a -> c -> c) -> c -> [a] -> c
\end{haskell}
以下是一个可能的实现:
\begin{haskell}
foldr step init = \as ->
  case as of
    [] -> init
    a : as -> step a (foldr step init as)
\end{haskell}

作为示例,我们可以使用\hask{foldr}计算自然数列表的元素之和:
\begin{haskell}
sum :: [Nat] -> Nat
sum = foldr plus Z
\end{haskell}


\begin{exercise}
考虑在列表定义中将$a$替换为终端对象时会发生什么。提示:自然数的基一编码是什么?
\end{exercise}
\begin{exercise}
有多少映射$h \colon L_a \to 1 + a$?我们可以使用列表递归器获得所有这些映射吗?Haskell函数的签名如何:
\begin{haskell}
h :: [a] -> Maybe a
\end{haskell}
\end{exercise}
\begin{exercise}
实现一个函数,如果列表足够长,则从列表中提取第三个元素。提示:使用\hask{Maybe a}作为结果类型。
\end{exercise}

\section{函子性}

函子性大致意味着能够转换数据结构的“内容”。列表$L_a$的内容是类型$a$。给定一个箭头$f \colon a \to b$,我们需要定义列表的映射$h \colon L_a \to L_b$。

列表由映射出属性定义,因此让我们将消除规则的目标$c$替换为$L_b$。我们得到:

\[
 \begin{tikzcd}
 1
 \arrow[r, "\text{Nil}_a"]
 \arrow[rd, "\mathit{init}"']
 &L_a
\arrow[d, dashed, "h"]
&a \times L_a
  \arrow[l, "\text{Cons}_a"']
\arrow[d, dashed, "id_a \times h"]
\\
& L_b
& a \times L_b
\arrow[l, "\mathit{step}"]
  \end{tikzcd}
\]
由于我们在这里处理两个不同的列表,我们必须区分它们的构造函数。例如,我们有:
\begin{align*}
\text{Nil}_a &\colon 1 \to L_a \\
\text{Nil}_b &\colon 1 \to L_b 
\end{align*}
$\text{Cons}$也是如此。

$\mathit{init}$的唯一候选是$\text{Nil}_b$,这意味着$h$作用于$a$的空列表时生成$b$的空列表:
\[ h \circ \text{Nil}_a = \text{Nil}_b \]

剩下的就是定义箭头:
\[\mathit{step} \colon a \times L_b \to L_b\]
我们可以猜测:
\[ \mathit{step} = \text{Cons}_b \circ (f \times id_{L_b}) \]

这对应于Haskell函数:

\begin{haskell}
mapList :: (a -> b) -> List a -> List b
mapList f = recList init step
  where
    init = Nil
    step (a, bs) = Cons (f a, bs)
\end{haskell}
或者,使用内置的列表语法并内联递归器,
\begin{haskell}
map :: (a -> b) -> [a] -> [b]
map f [] = []
map f (a : as) = f a : map f as
\end{haskell}

你可能会想知道是什么阻止我们选择$\mathit{step} = \mathit{snd}$,从而导致:
\begin{haskell}
badMap :: (a -> b) -> [a] -> [b]
badMap f [] = []
badMap f (a : as) = badMap f as
\end{haskell}
我们将在下一章中看到为什么这是一个糟糕的选择。(提示:当我们对$id$应用\hask{badMap}时会发生什么?)

\end{document}