\documentclass[DaoFP]{subfiles}
\begin{document}
\setcounter{chapter}{2}


\chapter{同构}

当我们说:
\[f \circ (g \circ h) = (f \circ g) \circ h \]
或:
\[ f = f \circ id \]
我们是在断言箭头的\emph{相等性}。左边的箭头是一个操作的结果,右边的箭头是另一个操作的结果。但结果是\emph{相等}的。

我们经常通过绘制\emph{交换}图来说明这种相等性,例如:

\[
 \begin{tikzcd}
 a
 \arrow[r, "h"]
 \arrow[rr, bend left=45, "g \circ h"]
 \arrow[rrr, bend left=80, "f \circ (g \circ h)"]
 \arrow[rrr, bend right=80, "(f \circ g) \circ h"]
 & b
 \arrow[r, "g"]
 \arrow[rr, bend right=45, "f \circ g"]
 &c
 \arrow[r, "f"]
 &d
 \end{tikzcd}
 \begin{tikzcd}
 a
 \arrow[r, "f"]
 \arrow[r, loop, "id"']
 &b
 \end{tikzcd}
\]

因此,我们通过比较箭头来判断它们是否相等。

我们\emph{不}比较对象的相等性\footnote{半开玩笑地说,在范畴论中,对象的相等性被认为是“邪恶”的}。我们将对象视为箭头的交汇点,因此如果我们想比较两个对象,我们会观察箭头。

\section{同构对象}

两个对象之间最简单的关系是一个箭头。

最简单的往返是两个方向相反的箭头的组合。

\[
 \begin{tikzcd}
 a
 \arrow[r, bend left, "f"]
 & b
 \arrow[l, bend left, "g"]
 \end{tikzcd}
\]

有两种可能的往返。一种是$g \circ f$,它从$a$到$a$。另一种是$f \circ g$,它从$b$到$b$。

如果它们都导致恒等箭头,那么我们说$g$是$f$的\emph{逆}
\[ g \circ f = id_a\]
\[f \circ g = id_b\]
我们将其记为$g = f^{-1}$(读作$f$的\emph{逆})。箭头$ f^{-1}$撤销了箭头$f$的作用。

这样的一对箭头称为\emph{同构},两个对象称为\emph{同构的}。

在编程中,两个同构的类型具有相同的外部行为。一个类型可以用另一个类型来实现,反之亦然。一个类型可以被另一个类型替换,而不会改变系统的行为(除了可能的性能)。

在Haskell中,我们经常根据其他类型定义类型。如果只是将一个名称替换为另一个名称,我们使用类型同义词。例如:
\begin{haskell}
type MyTemperature = Int
\end{haskell}
这让我们在涉及天气的程序中使用\hask{MyTemperature}作为整数的更具描述性的名称。这两个类型是相等的——在所有上下文中,一个可以替换为另一个,所有接受\hask{Int}参数的函数都可以用\hask{MyTemperature}调用。

当我们想要定义一个与现有类型\emph{同构}但不相等的新类型时,我们可以使用\index{newtype}\hask{newtype}。例如,我们可以将\hask{Temperature}定义为:
\begin{haskell}
newtype Temperature = Temp Int
\end{haskell}
这里,\hask{Temperature}是新类型,\hask{Temp}是一个\emph{数据构造函数}。数据构造函数只是一个函数\footnote{与常规函数不同,数据构造函数的名称必须以大写字母开头}:
\begin{haskell}
Temp :: Int -> Temp
\end{haskell}
我们还可以定义它的逆:
\begin{haskell}
toInt :: Temperature -> Int
toInt (Temp i) = i
\end{haskell}
从而完成同构:
\[
 \begin{tikzcd}
 \hask{Int}
 \arrow[r, bend left, "\hask{Temp}"]
 & \hask{Temperature}
 \arrow[l, bend left, "\hask{toInt}"]
 \end{tikzcd}
\]
由于这是一个常见的构造,Haskell提供了特殊语法,可以同时定义这两个函数:
\begin{haskell}
newtype Temperature = Temp { toInt :: Int}
\end{haskell}

因为\hask{Temperature}是一个新类型,操作整数的函数不会接受它作为参数。这样,程序员可以限制它的使用方式。我们仍然可以使用同构来有选择地允许某些操作,例如在函数\hask{negate :: Int -> Int}的应用中。执行此操作的代码是图的直接翻译:
\[ \hask{Temperature} \xrightarrow{\hask{toInt}} \hask{Int} \xrightarrow{\hask{negate}} \hask{Int} \xrightarrow{\hask{Temp}} \hask{Temperature} \]
\begin{haskell}
invert :: Temperature -> Temperature
invert = Temp . negate . toInt
\end{haskell}


\subsection{同构与双射}

同构的存在告诉我们关于它连接的两个对象的什么信息?

我们说过,对象通过它们与其他对象的交互来描述。因此,让我们从观察者$x$的角度考虑这两个同构对象的样子。取一个从$x$到$a$的箭头$h$。

\[
 \begin{tikzcd}
 & x
 \arrow[ld, red, "h"']
 \\
 a
 \arrow[rr, "f"]
  && b
 \arrow[ll, bend left,  "f^{-1}"]
 \end{tikzcd}
\]
有一个对应的从$x$到$b$的箭头。它只是$f \circ h$的组合,或者$(f \circ -)$对$h$的作用。
\[
 \begin{tikzcd}
 & x
 \arrow[ld, "h"']
 \arrow[rd, red, "f \circ h"]
 \\
 a
 \arrow[rr, "f"]
  && b
 \arrow[ll, bend left,  "f^{-1}"]
 \end{tikzcd}
\]
类似地,对于任何探测$b$的箭头,都有一个对应的探测$a$的箭头。它由$(f^{-1} \circ -)$的作用给出。

我们可以使用映射$(f \circ -)$和$(f^{-1} \circ -)$在$a$和$b$之间来回移动焦点。

我们可以将这两个映射组合起来(见练习\ref{ex-yoneda-composition})形成一个往返。结果与如果我们应用复合$((f^{-1} \circ f) \circ -)$相同。但这等于$(id_a \circ  -)$,正如我们从练习\ref{ex-yoneda-identity}中所知,它不会改变箭头。

类似地,由$f \circ f^{-1}$引起的往返不会改变箭头$x \to b$。

这在这两组箭头之间创建了一个“伙伴系统”。想象每个箭头向其伙伴发送一条消息,由$f$或$f^{-1}$决定。然后,每个箭头将恰好收到一条消息,并且这条消息将来自其伙伴。没有箭头会被遗漏,也没有箭头会收到多条消息。数学家称这种伙伴系统为\emph{双射}或一一对应。

因此,从$x$的角度来看,两个对象$a$和$b$在箭头上看起来完全相同。在箭头上,这两个对象之间没有区别。

两个同构的对象具有完全相同的性质。

特别是,如果你将$x$替换为终端对象$1$,你会看到这两个对象具有相同的元素。对于每个元素$x \colon 1 \to a$,都有一个对应的元素$y \colon 1 \to b$,即$y = f \circ x$,反之亦然。同构对象的元素之间存在双射。

这种无法区分的对象被称为\emph{同构的},因为它们具有“相同的形状”。你见过一个,你就见过所有。

我们将这种同构写为:

\[a \cong b\]

在处理对象时,我们使用同构来代替相等。

在经典逻辑中,如果B从A推出,A从B推出,那么A和B在逻辑上是等价的。我们经常说B为真“当且仅当”A为真。然而,与逻辑和类型理论之间的先前类比不同,如果你认为证明是相关的,这个类比并不那么直接。事实上,它导致了基础数学的一个新分支的发展,称为同伦类型理论,简称HoTT。

\begin{exercise}
论证从两个同构对象\emph{出发}的箭头之间存在双射。绘制相应的图。
\end{exercise}


\begin{exercise}
证明每个对象都与自身同构。
\end{exercise}

\begin{exercise}
如果有两个终端对象,证明它们是同构的。
\end{exercise}

\begin{exercise}
证明前一个练习中的同构是唯一的。
\end{exercise}

\section{自然性}

我们已经看到,当两个对象同构时,我们可以使用后复合在它们之间切换焦点:$(f \circ -)$或$(f^{-1} \circ -)$。

相反,要在不同的观察者之间切换,我们将使用前复合。

确实,从$x$探测$a$的箭头$h$与从$y$探测同一对象的箭头$h\circ g$相关。

\[
 \begin{tikzcd}
 x
 \arrow[d, "h"']
 && y
 \arrow[ll, dashed, "g"']
  \arrow[dll, red, "h \circ g"']
 \\
 a
 \arrow[rr, "f"]
  && b
 \arrow[ll, bend left,  "f^{-1}"]
 \end{tikzcd}
\]
类似地,从$x$探测$b$的箭头$h'$对应于从$y$探测它的箭头$h' \circ g$。

\[
 \begin{tikzcd}
 x
 \arrow[drr, "h'"]
 && y
 \arrow[ll, dashed, "g"']
  \arrow[d, red, "h' \circ g"]
 \\
 a
 \arrow[rr, "f"]
  && b
 \arrow[ll, bend left,  "f^{-1}"]
 \end{tikzcd}
\]
在这两种情况下,我们通过应用前复合$(- \circ g)$将视角从$x$切换到$y$。

重要的观察是,视角的变化保留了由同构建立的伙伴系统。如果两个箭头从$x$的角度是伙伴,那么它们从$y$的角度仍然是伙伴。这就像说,先与$g$前复合(切换视角)再与$f$后复合(切换焦点),或者先与$f$后复合再与$g$前复合,结果是一样的。符号上,我们将其写为:

\[(- \circ g) \circ (f \circ -) = (f \circ -) \circ (- \circ g)\]
我们称之为\emph{自然性}条件。

这个方程的意义在你将其应用于态射$h \colon x \to a$时显现出来。两边都计算为$f \circ h \circ g$。
\[
 \begin{tikzcd}
 h
 \arrow[r, mapsto, "(- \circ g)"]
 \arrow[d, mapsto, "(f \circ -)"']
 & h \circ g
 \arrow[d, mapsto, "(f \circ -)"]
 \\
 f \circ h
 \arrow[r, mapsto, "(- \circ g)"]
& f \circ h \circ g
 \end{tikzcd}
\]

在这里,由于结合性,自然性条件自动满足,但我们很快就会看到它在更不平凡的情况下被推广。

箭头用于广播关于同构的信息。自然性告诉我们,所有对象都能获得一致的观点,与路径无关。

我们还可以反转观察者和被观察者的角色。例如,使用箭头$h \colon a \to x$,对象$a$可以探测任意对象$x$。如果存在箭头$g \colon x \to y$,它可以将焦点切换到$y$。通过前复合$f^{-1}$将视角切换到$b$。
\[
 \begin{tikzcd}
 a
 \arrow[rr, "f"]
 \arrow[d, "h"']
 \arrow[rrd, red, "g\circ h"]
 && b
  \arrow[ll, bend right,  "f^{-1}"']
 \\
 x
 \arrow[rr, dashed, "g"']
  && y
 \end{tikzcd}
\]
再次,我们有了自然性条件,这次是从同构对的角度:
\[(- \circ f^{-1}) \circ (g \circ -) = (g \circ -) \circ (- \circ f^{-1}) \]

这种需要采取两个步骤从一个地方移动到另一个地方的情况在范畴论中是典型的。在这里,前复合和后复合的操作可以以任何顺序进行——我们说它们\index{交换操作}\emph{交换}。但通常我们采取步骤的顺序会导致不同的结果。我们经常施加交换条件,并说如果一个操作与另一个操作兼容,则这些条件成立。

\begin{exercise}
证明$f^{-1}$的自然性条件的两边,当作用于$h$时,简化为:
\[
 \begin{tikzcd}
 b \arrow[r, "f^{-1}"] &a \arrow[r, "h"] & x \arrow[r, "g"] & y
\end{tikzcd}
\]

\end{exercise}

\section{用箭头推理}

Yoneda大师说:“看箭头!”

如果两个对象同构,它们具有相同的入箭头集。

如果两个对象同构,它们也具有相同的出箭头集。

如果你想看两个对象是否同构,看箭头!

\medskip

当两个对象$a$和$b$同构时,任何同构$f$都会在相应的箭头集之间诱导出一一映射$(f \circ -)$。  
\[
 \begin{tikzcd}
 \node(x) at (0, 2) {x};
 \node(a) at (-2, 0) {a};
 \node(b) at (2, 0) {b};
 \node(c1) at (-1, 1.5) {};
 \node(c2) at (-1.5, 1) {};
 \node(c3) at (-1, 2) {};
 \node(c4) at (-2, 1) {};
 \node(d1) at (1, 1.5) {};
 \node(d2) at (1.5, 1) {};
 \node(d3) at (1, 2) {};
 \node(d4) at (2, 1) {};
\node (aa) at (-1, 0.75) {};
 \node (bb) at (1, 0.75) {};
 \draw[->] (x) .. controls (c1)  and (c2) .. (a); % bend
 \draw[->, green] (x) .. controls (c3)  and (c4) .. (a); % bend
 \draw[->, blue] (x) -- (a); 
  \draw[->] (x) .. controls (d1)  and (d2) .. (b); % bend
 \draw[->, green] (x) .. controls (d3)  and (d4) .. (b); % bend
 \draw[->, blue] (x) -- (b); 
 \draw[->, red, dashed] (aa) -- node[above]{(f \circ -)} (bb);
\draw[->] (a) -- node[below]{f} (b);
 \end{tikzcd}
\]
函数$(f \circ -)$将每个箭头$h \colon x \to a$映射到箭头$f \circ h \colon x \to b$。它的逆$(f^{-1} \circ -)$将每个箭头$h' \colon x \to b$映射到箭头$(f^{-1} \circ h')$。

假设我们不知道对象是否同构,但我们知道存在一个可逆映射$\alpha_x$,它在每个对象$x$的箭头集之间建立了一一对应关系。换句话说,对于每个$x$,$\alpha_x$是箭头的双射。 
\[
 \begin{tikzcd}
 \node(x) at (0, 2) {x};
 \node(a) at (-2, 0) {a};
 \node(b) at (2, 0) {b};
 \node(c1) at (-1, 1.5) {};
 \node(c2) at (-1.5, 1) {};
 \node(c3) at (-1, 2) {};
 \node(c4) at (-2, 1) {};
 \node(d1) at (1, 1.5) {};
 \node(d2) at (1.5, 1) {};
 \node(d3) at (1, 2) {};
 \node(d4) at (2, 1) {};
\node (aa) at (-1, 0.75) {};
 \node (bb) at (1, 0.75) {};
 \draw[->] (x) .. controls (c1)  and (c2) .. (a); % bend
 \draw[->, green] (x) .. controls (c3)  and (c4) .. (a); % bend
 \draw[->, blue] (x) -- (a); 
  \draw[->] (x) .. controls (d1)  and (d2) .. (b); % bend
 \draw[->, green] (x) .. controls (d3)  and (d4) .. (b); % bend
 \draw[->, blue] (x) -- (b); 
 \draw[->, red, dashed] (aa) -- node[above]{\alpha_x} (bb);
 \end{tikzcd}
\]
之前,箭头的双射是由同构$f$生成的。现在,箭头的双射由$\alpha_x$给出。这是否意味着这两个对象是同构的?我们可以从映射族$\alpha_x$中构造出同构$f$吗?答案是“是的”,只要映射族$\alpha_x$满足自然性条件。

这是$\alpha_x$在特定箭头$h$上的作用。 
\[
 \begin{tikzcd}
 x
 \arrow[d, "h"']
 \arrow[rrd, red, "\alpha_x h"]
  \\
 a
  && b
 \end{tikzcd}
\]
这个映射,连同它的逆$\alpha^{-1}_x$,它将箭头$x \to b$映射到箭头$x \to a$,将扮演$(f \circ -)$和$(f^{-1} \circ -)$的角色,如果确实存在同构$f$。映射族$\alpha$描述了一种“人工”的从$a$切换到$b$的方式。

这是从另一个观察者$y$的角度看的相同情况:
\[
 \begin{tikzcd}
 x
  && y
 \arrow[lld, "h'"']
 \arrow[d, red, "\alpha_y h'"]
 \\
 a
  && b
 \end{tikzcd}
\]
注意,$y$使用的是同一族中的不同映射$\alpha_y$。

这两个映射,$\alpha_x$和$\alpha_y$,每当存在态射$g \colon y \to x$时就会纠缠在一起。在这种情况下,与$g$的前复合允许我们从$x$切换到$y$的视角(注意方向)

\[
 \begin{tikzcd}
 x
 \arrow[d, "h"']
 && y
 \arrow[ll, dashed, "g"']
 \arrow[lld, red, "h \circ g"]
 \\
 a
  && b
 \end{tikzcd}
\]
我们将焦点的切换与视角的切换分开了。前者由$\alpha$完成,后者由前复合完成。自然性在这两者之间施加了兼容性条件。

确实,从某个$h$开始,我们可以先应用$(- \circ g)$切换到$y$的视角,然后应用$\alpha_y$将焦点切换到$b$:
\[ \alpha_y \circ (- \circ g) \]
或者我们可以先让$x$使用$\alpha_x$将焦点切换到$b$,然后使用$(- \circ g)$切换视角:
\[ (- \circ g) \circ \alpha_x \]
在这两种情况下,我们最终都会从$y$的角度看$b$。我们之前做过这个练习,当$a$和$b$之间存在同构时,我们发现结果是相同的。我们称之为自然性条件。

如果我们希望$\alpha$给我们一个同构,我们必须施加等效的自然性条件:
\[ \alpha_y \circ (- \circ g) = (- \circ g) \circ \alpha_x \]
当作用于某个箭头$h \colon x \to a$时,我们希望这个图交换:
\[
 \begin{tikzcd}
 h
 \arrow[r, mapsto, "(- \circ g)"]
 \arrow[d, mapsto, red, "\alpha_x"]
 & h \circ g
 \arrow[d, mapsto, red, "\alpha_y"]
 \\
 \alpha_x h
 \arrow[r, mapsto, "(- \circ g)"]
&(\alpha_x h) \circ g = \alpha_y (h \circ g)
 \end{tikzcd}
\]
这样我们就知道,用$(f \circ -)$替换所有$\alpha$会起作用。但这样的$f$存在吗?我们可以从$\alpha$中重建$f$吗?答案是肯定的,我们将使用Yoneda技巧来实现这一点。

由于$\alpha_x$是为任何对象$x$定义的,它也为$a$本身定义。根据定义,$\alpha_a$将态射$a \to a$映射到态射$a \to b$。我们肯定知道至少存在一个态射$a \to a$,即恒等态射$id_a$。事实证明,我们寻找的同构$f$由下式给出:
\[f = \alpha_a (id_a)\]
或者,图示为:
\[
 \begin{tikzcd}
a
 \arrow[d, "id_a"']
 \arrow[rrd, red, "f = \alpha_a (id_a)"]
  \\
 a
  && b
 \end{tikzcd}
\]

让我们验证这一点。如果$f$确实是我们的同构,那么对于任何$x$,$\alpha_x$应该等于$(f \circ -)$。为了看到这一点,让我们将自然性条件中的$x$替换为$a$。我们得到:
\[\alpha_y(h \circ g) = (\alpha_a h) \circ g \]
如下图所示:
\[
 \begin{tikzcd}
 a
 \arrow[d, "h"']
 \arrow[rrd,  red, "\alpha_a (h)"']
 && y
 \arrow[ll, "g"']
 \arrow[d, red, "\alpha_y (h \circ g)"]
   \\
 a
  && b
 \end{tikzcd}
\]


由于$h$的源和目标都是$a$,这个等式对于$h = id_a$也必须成立
\[\alpha_y (id_a \circ g) = (\alpha_a (id_a)) \circ g \]
但$id_a \circ g$等于$g$,$\alpha_a(id_a)$是我们的$f$,所以我们得到:
\[\alpha_y g = f \circ g = (f \circ -) g\]
换句话说,对于每个对象$y$和每个态射$g \colon y \to a$,$\alpha_y = (f \circ -)$。

注意,即使$\alpha_x$是为每个$x$和每个箭头$x \to a$单独定义的,它最终完全由其在单个恒等箭头上的值决定。这就是自然性的力量!
\subsection{反转箭头}
正如老子所说,观察者与被观察者之间的二元性只有在观察者被允许与被观察者交换角色时才能完全成立。

再次,我们想证明两个对象$a$和$b$是同构的,但这次我们想将它们视为观察者。箭头$h \colon a \to x$从$a$的角度探测任意对象$x$。之前,当我们知道这两个对象是同构的时,我们能够使用$(- \circ f^{-1})$将视角切换到$b$。这次我们手头有一个变换$\beta_x$。它在箭头上建立了一一对应关系。
\[
 \begin{tikzcd}
 x
 \\
 a
\arrow[u, "h"]
 && b
  \arrow[llu, red, "\beta_x h"']
  \end{tikzcd}
\]
如果我们想观察另一个对象$y$,我们将使用$\beta_y$在$a$和$b$之间切换视角,依此类推。

\[
 \begin{tikzcd}
 \node(x) at (0, 2) {x};
 \node(a) at (-2, 0) {a};
 \node(b) at (2, 0) {b};
 \node(c1) at (-1, 1.5) {};
 \node(c2) at (-1.5, 1) {};
 \node(c3) at (-1, 2) {};
 \node(c4) at (-2, 1) {};
 \node(d1) at (1, 1.5) {};
 \node(d2) at (1.5, 1) {};
 \node(d3) at (1, 2) {};
 \node(d4) at (2, 1) {};
\node (aa) at (-1, 0.75) {};
 \node (bb) at (1, 0.75) {};
 \draw[<-] (x) .. controls (c1)  and (c2) .. (a); % bend
 \draw[<-, green] (x) .. controls (c3)  and (c4) .. (a); % bend
 \draw[<-, blue] (x) -- (a); 
  \draw[<-] (x) .. controls (d1)  and (d2) .. (b); % bend
 \draw[<-, green] (x) .. controls (d3)  and (d4) .. (b); % bend
 \draw[<-, blue] (x) -- (b); 
 \draw[->, red, dashed] (aa) -- node[above]{\beta_x} (bb);
 \end{tikzcd}
\]


如果两个对象$x$和$y$通过箭头$g \colon x \to y$连接,那么我们还可以选择使用$(g \circ -)$切换焦点。如果我们想同时做两件事:切换视角和切换焦点,有两种方法。自然性要求结果相等:
\[ (g \circ -) \circ \beta_x = \beta_y \circ (g \circ -) \]
确实,如果我们将$\beta$替换为$(- \circ f^{-1})$,我们会恢复同构的自然性条件。

\begin{exercise}
使用恒等态的技巧从映射族$\beta$中恢复$f^{-1}$。
\end{exercise}
\begin{exercise}
使用前一个练习中的$f^{-1}$,评估任意对象$y$和任意箭头$g \colon a \to y$的$\beta_y g$。
\end{exercise}


正如老子所说:要证明同构,通常更容易在万箭之间定义自然变换,而不是在两个对象之间找到一对箭头。

\end{document}