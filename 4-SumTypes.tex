\documentclass[DaoFP]{subfiles}
\begin{document}
\setcounter{chapter}{3}

\chapter{和类型(Sum Types)}

\section{布尔类型(Bool)}

我们已经知道如何组合箭头。但是如何组合对象呢?

我们已经定义了$0$(初始对象)和$1$(终止对象)。那么$2$如果不是$1$加$1$,又是什么呢?

一个$2$是一个具有两个元素的对象:两个从$1$出发的箭头。让我们称其中一个箭头为\hask{True},另一个为\hask{False}。不要将这些名称与初始对象和终止对象的逻辑解释混淆。这两个是\emph{箭头}。

\[
 \begin{tikzcd}
 1
 \arrow[dd, bend right, "\text{True}"']
 \arrow[dd, bend left, "\text{False}"]
 \\
 \\
2
 \end{tikzcd}
\]

这个简单的想法可以立即在Haskell\footnote{这种定义风格在Haskell中被称为广义代数数据类型(Generalized Algebraic Data Types)或\hask{GADTs}}中表达为一个类型的定义,传统上称为\hask{Bool},以其发明者乔治·布尔(George Boole, 1815-1864)命名。

\begin{haskell}
data Bool where
  True  :: () -> Bool
  False :: () -> Bool
\end{haskell}
它对应于相同的图表(只是有一些Haskell的重命名):
\[
 \begin{tikzcd}
 \hask{()}
 \arrow[dd, bend right, "\hask{True}"']
 \arrow[dd, bend left, "\hask{False}"]
 \\
 \\
\hask{Bool}
 \end{tikzcd}
\]

正如我们之前所见,元素有一个简写符号,所以这里是一个更紧凑的版本:

\begin{haskell}
data Bool where
  True  :: Bool
  False :: Bool
\end{haskell}

我们现在可以定义一个类型为\hask{Bool}的项,例如
\begin{haskell}
x :: Bool
x = True
\end{haskell}
第一行声明\hask{x}是\hask{Bool}的一个元素(实际上是一个函数\hask{()->Bool}),第二行告诉我们它是两个中的哪一个。

我们在定义\hask{Bool}时使用的函数\hask{True}和\hask{False}被称为\index{数据构造器}\emph{数据构造器}。它们可以用来构造特定的项,如上面的例子。顺便说一下,在Haskell中,函数名以小写字母开头,除非它们是数据构造器。

我们对类型\hask{Bool}的定义仍然不完整。我们知道如何构造一个\hask{Bool}项,但我们不知道如何处理它。我们必须能够定义从\hask{Bool}出发的箭头——\index{映射出}\emph{映射出} \hask{Bool}。

第一个观察是,如果我们有一个从\hask{Bool}到某个具体类型\hask{A}的箭头\hask{h},那么我们只需通过组合就可以自动得到两个从单位到\hask{A}的箭头\hask{x}和\hask{y}。以下两个(变形的)三角形是可交换的:

\[
 \begin{tikzcd}
 \hask{()}
 \arrow[dd, bend right, "\text{True}"']
 \arrow[dd, bend left, "\text{False}"]
  \arrow[ddd, bend right =90, "x"']
 \arrow[ddd, bend left=90, "y"]
\\
 \\
\text{Bool}
\arrow[d, dashed, "h"]
\\
A
 \end{tikzcd}
\]
换句话说,每个函数\hask{Bool->A}都会产生一对\hask{A}的元素。

给定一个具体类型\hask{A}:
\begin{haskell}
h :: Bool -> A
\end{haskell}
我们有:
\begin{haskell}
x = h True
y = h False
\end{haskell}
其中
\begin{haskell}
x :: A
y :: A
\end{haskell}
注意使用简写符号表示函数对元素的应用:
\begin{haskell}
h True -- 意思是:h . True
\end{haskell}

我们现在准备通过添加一个条件来完成\hask{Bool}的定义,即任何从\hask{Bool}到\hask{A}的函数不仅产生而且\emph{等价于}一对\hask{A}的元素。换句话说,一对元素唯一地确定了一个从\hask{Bool}出发的函数。

这意味着我们可以用两种方式解释上面的图表:给定\hask{h},我们可以很容易地得到\hask{x}和\hask{y}。但反过来也是成立的:一对元素\hask{x}和\hask{y}唯一地\emph{定义}了\hask{h}。

我们这里有一个双射在起作用。这次是一对元素$(x, y)$和一个箭头$h$之间的一一映射。

在Haskell中,\hask{h}的这个定义封装在\index{\hask{if}语句}\hask{if}、\hask{then}、\hask{else}构造中。给定
\begin{haskell}
x :: A
y :: A
\end{haskell}
我们定义映射出
\begin{haskell}
h :: Bool -> A
h b = if b then x else y
\end{haskell}
这里,\hask{b}是类型\hask{Bool}的一个项。

一般来说,数据类型使用\index{引入规则}\emph{引入}规则创建,并使用\index{消除规则}\emph{消除}规则解构。\hask{Bool}数据类型有两个引入规则,一个使用\hask{True},另一个使用\hask{False}。\hask{if}、\hask{then}、\hask{else}构造定义了消除规则。

给定上述\hask{h}的定义,我们可以检索用于定义它的两个项,这被称为\index{计算规则}\emph{计算}规则。它告诉我们如何计算\hask{h}的结果。如果我们用\hask{True}调用\hask{h},结果是\hask{x};如果我们用\hask{False}调用它,结果是\hask{y}。

我们永远不应忘记编程的目的:将复杂问题分解为一系列更简单的问题。\hask{Bool}的定义说明了这个想法。每当我们必须构造一个从\hask{Bool}出发的映射时,我们将其分解为构造目标类型的一对元素的两个较小任务。我们将一个较大的问题换成了两个较简单的问题。

\subsection{示例}

让我们做一些例子。我们还没有定义很多类型,所以我们将仅限于从\hask{Bool}到\hask{Void}、\hask{()}或\hask{Bool}的映射。然而,这些边缘情况可能会为众所周知的结果提供新的见解。

我们已经决定,除了恒等函数外,不能有以\hask{Void}为目标的函数,所以我们不期望有任何从\hask{Bool}到\hask{Void}的函数。确实,我们有零对\hask{Void}的元素。

那么从\hask{Bool}到\hask{()}的函数呢?由于\hask{()}是终止的,所以只能有一个从\hask{Bool}到它的函数。确实,这个函数对应于从\hask{()}到\hask{()}的单一可能函数对——两者都是恒等函数。到目前为止,一切顺利。
\[
 \begin{tikzcd}
 \hask{()}
 \arrow[dd, bend right, "\text{True}"']
 \arrow[dd, bend left, "\text{False}"]
  \arrow[ddd, bend right =90, "\hask{id}"']
 \arrow[ddd, bend left=90, "\hask{id}"]
\\
 \\
\text{Bool}
\arrow[d, dashed, "h"]
\\
\hask{()}
 \end{tikzcd}
\]

有趣的情况是从\hask{Bool}到\hask{Bool}的函数。让我们将\hask{Bool}代入\hask{A}的位置:
\[
 \begin{tikzcd}
 \hask{()}
 \arrow[dd, bend right, "\text{True}"']
 \arrow[dd, bend left, "\text{False}"]
  \arrow[ddd, bend right =90, "x"']
 \arrow[ddd, bend left=90, "y"]
\\
 \\
\text{Bool}
\arrow[d, dashed, "h"]
\\
\text{Bool}
 \end{tikzcd}
\]
我们有多少对从\hask{()}到\hask{Bool}的函数$(x, y)$可供使用?只有两个这样的函数,\hask{True}和\hask{False},所以我们可以形成四对。这些是$(True, True)$、$(False, False)$、$(True, False)$和$(False, True)$。因此,只能有四个从\hask{Bool}到\hask{Bool}的函数。

我们可以使用\hask{if}、\hask{then}、\hask{else}构造在Haskell中编写它们。例如,最后一个,我们称之为\hask{not},定义如下:
\begin{haskell}
not :: Bool -> Bool
not b = if b then False else True
\end{haskell}

我们还可以将从\hask{Bool}到\hask{A}的函数视为箭头对象的元素,或指数对象$A^2$,其中$2$是\hask{Bool}对象。根据我们的计数,$0^2$中有零个元素,$1^2$中有一个元素,$2^2$中有四个元素。这正是我们从高中代数中所期望的,其中数字实际上意味着数字。

\begin{exercise}
编写其他三个函数\hask{Bool->Bool}的实现。
\end{exercise}

\section{枚举类型}

在0、1、2之后是什么?一个具有三个数据构造子的对象。例如:
\begin{haskell}
data RGB where
  Red   :: RGB
  Green :: RGB
  Blue  :: RGB
\end{haskell}
如果你厌倦了冗余的语法,这种定义有一个简写形式:

\begin{haskell}
data RGB = Red | Green | Blue
\end{haskell}
这个引入规则允许我们构造类型为\hask{RGB}的项,例如:
\begin{haskell}
c :: RGB
c = Blue
\end{haskell}
为了定义从\hask{RGB}出发的映射,我们需要一个更一般的消去模式。就像从\hask{Bool}出发的函数由两个元素决定一样,从\hask{RGB}到\hask{A}的函数由\hask{A}的三个元素\hask{x}、\hask{y}和\hask{z}决定。我们使用\emph{模式匹配}语法来编写这样的函数:
\begin{haskell}
h :: RGB -> A
h Red   = x
h Green = y
h Blue  = z
\end{haskell}
这只是一个函数,其定义被分成了三种情况。

对于\hask{Bool},也可以使用相同的语法来代替\hask{if}、\hask{then}、\hask{else}:
\begin{haskell}
h :: Bool -> A
h True  = x
h False = y
\end{haskell}
事实上,还有第三种方式可以使用\index{\hask{case}语句}\hask{case}语句来编写相同的内容:
\begin{haskell}
h c = case c of
  Red   -> x
  Green -> y
  Blue  -> z
\end{haskell}
或者甚至
\begin{haskell}
h :: Bool -> A
h b = case b of
  True  -> x
  False -> y
\end{haskell}
在编程时,你可以根据方便使用这些方式中的任何一种。

这些模式也适用于具有四个、五个或更多数据构造子的类型。例如,一个十进制数字是以下之一:
\begin{haskell}
data Digit = Zero | One | Two | Three | ... | Nine
\end{haskell}

有一个巨大的Unicode字符枚举类型叫做\hask{Char}。它们的构造子有特殊的名称:你可以在两个单引号之间写入字符本身,例如:
\begin{haskell}
c :: Char
c = 'a'
\end{haskell}

正如老子所说,万物的模式需要许多年才能完成,因此人们想出了通配符模式,即下划线,它可以匹配任何内容。

因为模式是按顺序匹配的,你应该将通配符模式作为一系列模式中的最后一个:
\begin{haskell}
yesno :: Char -> Bool
yesno c = case c of
  'y' -> True
  'Y' -> True
  _   -> False
\end{haskell}

但我们为什么要止步于此呢?类型\hask{Int}可以被视为在$-2^{29}$到$2^{29}$之间(或更多,取决于实现)的整数枚举。当然,对这样的范围进行穷举模式匹配是不可能的,但原则仍然成立。

在实践中,用于Unicode字符的\hask{Char}类型、用于固定精度整数的\hask{Int}类型、用于双精度浮点数的\hask{Double}类型以及其他几种类型,都是内置在语言中的。

这些不是无限类型。它们的元素可以被枚举,即使这可能需要一万年。不过,类型\hask{Integer}是无限的。

\section{和类型(Sum Types)}

\hask{Bool} 类型可以被看作是 $2 = 1 + 1$ 的和。但没有什么能阻止我们用另一个类型替换 $1$,甚至用不同的类型替换每个 $1$。我们可以通过使用两个箭头来定义一个新的类型 $a + b$。我们称它们为 \hask{Left} 和 \hask{Right}。定义图是引入规则:
\[
 \begin{tikzcd}
 a
 \arrow[dr,  "\text{Left}"']
 && b
 \arrow[dl, "\text{Right}"]
 \\
&a + b
 \end{tikzcd}
\]
在 Haskell 中,类型 $a + b$ 被称为 \hask{Either a b}。与 \hask{Bool} 类比,我们可以将其定义为
\begin{haskell}
data Either a b where
  Left  :: a -> Either a b
  Right :: b -> Either a b
\end{haskell}
(注意类型变量使用小写字母。)

类似地,从 $a + b$ 到某个类型 $c$ 的映射由以下交换图确定:
\[
 \begin{tikzcd}
 a
 \arrow[dr,  bend left, "\text{Left}"']
 \arrow[ddr, bend right, "f"']
 && b
 \arrow[dl, bend right, "\text{Right}"]
 \arrow[ddl, bend left, "g"]
 \\
&a + b
\arrow[d, dashed, "h"]
\\
& c
 \end{tikzcd}
\]
给定一个函数 $h$,我们只需将其与 \hask{Left} 和 \hask{Right} 组合,就可以得到一对函数 $f$ 和 $g$。反过来,这样的一对函数唯一地确定了 $h$。这是消除规则。

当我们将这个图转换为 Haskell 时,我们需要选择两个类型的元素。我们可以通过从终端对象定义箭头 $a$ 和 $b$ 来实现。
\[
 \begin{tikzcd}
 &1
 \arrow[ld, "a"']
 \arrow[rd, "b"]
 \\
 a
 \arrow[dr,  bend left, "\text{Left}"']
 \arrow[ddr, bend right, "f"']
 && b
 \arrow[dl, bend right, "\text{Right}"]
 \arrow[ddl, bend left, "g"]
 \\
&a + b
\arrow[d, dashed, "h"]
\\
& c
 \end{tikzcd}
\]
按照图中的箭头,我们得到:
\[h \circ \text{Left} \circ a = f \circ a\]
\[h \circ \text{Right} \circ b = g \circ b\]

Haskell 语法几乎逐字重复了这些等式,从而产生了用于定义 \hask{h} 的模式匹配语法:

\begin{haskell}
h :: Either a b -> c
h (Left  a) = f a
h (Right b) = g b
\end{haskell}
(再次注意,类型变量使用小写字母,并且相同字母用于该类型的项。与人类不同,编译器不会因此感到困惑。)

你也可以从右到左阅读这些等式,你会看到和类型的计算规则:用于定义 \hask{h} 的两个函数可以通过将 \hask{h} 应用于 \hask{(Left a)} 和 \hask{(Right b)} 来恢复。

你也可以使用 \hask{case} 语法来定义 \hask{h}:
\begin{haskell}
h e = case e of
  Left  a -> f a
  Right b -> g b
\end{haskell}

那么数据类型的本质是什么?它不过是操作箭头的配方。

\subsection{Maybe 类型}

\hask{Maybe} 是一个非常有用的数据类型,它被定义为任何 $a$ 的和 $1 + a$。这是它在 Haskell 中的定义:
\begin{haskell}
data Maybe a where
  Nothing :: () -> Maybe a
  Just    ::  a -> Maybe a
\end{haskell}
数据构造函数 \hask{Nothing} 是从单元类型出发的箭头,而 \hask{Just} 从 \hask{a} 构造 \hask{Maybe a}。\hask{Maybe a} 与 \hask{Either () a} 同构。它也可以使用简写符号定义为
\begin{haskell}
data Maybe a = Nothing | Just a
\end{haskell}

\hask{Maybe} 主要用于编码部分函数的返回类型:这些函数在某些参数值下未定义。在这种情况下,这些函数不会失败,而是返回 \hask{Nothing}。在其他编程语言中,部分函数通常通过异常(或核心转储)来实现。

\subsection{逻辑}

在逻辑中,命题 $A + B$ 被称为\index{alternative}择一(alternative),或\emph{逻辑或}。你可以通过提供 $A$ 的证明或 $B$ 的证明来证明它。其中任何一个都足够。

如果你想证明 $C$ 从 $A+B$ 得出,你必须为两种可能性做好准备:要么有人通过证明 $A$ 来证明 $A+B$(而 $B$ 可能为假),要么通过证明 $B$ 来证明 $A+B$(而 $A$ 可能为假)。在第一种情况下,你必须证明 $C$ 从 $A$ 得出。在第二种情况下,你需要一个证明 $C$ 从 $B$ 得出。这些正是 $A+B$ 消除规则中的箭头。

\section{余积范畴}

在Haskell中,我们可以使用\hask{Either}定义任意两种类型的和。一个范畴中如果所有和都存在,并且存在初始对象,则称为\emph{余积范畴},而和称为\emph{余积}。你可能已经注意到,和类型模仿了数字的加法。事实证明,初始对象扮演了零的角色。

\subsection{一加零}

我们首先证明$1 + 0 \cong 1$,即终端对象与初始对象的和同构于终端对象。这类证明的标准方法是使用Yoneda技巧。由于和类型是通过映射来定义的,我们应该比较从两边\emph{出来}的箭头。

Yoneda论证指出,如果两个对象到任意对象$a$的箭头集合之间存在双射$\beta_a$,并且这个双射是自然的,那么这两个对象是同构的。

让我们看看$1 + 0$的定义及其到任意对象$a$的映射。这个映射由一对$(x, \mbox{!`})$定义,其中$x$是$a$的一个元素,$\mbox{!`}$是从初始对象到$a$的唯一箭头(在Haskell中是\hask{absurd}函数)。

\[
 \begin{tikzcd}
 1
 \arrow[dr,  bend left, "\text{Left}"']
 \arrow[ddr, bend right, "x"']
 && 0
 \arrow[dl, bend right, "\text{Right}"]
 \arrow[ddl, bend left, "\mbox{!`}"]
 \\
&1 + 0
\arrow[d, dashed, "h"]
\\
& a
 \end{tikzcd}
 \qquad
 \begin{tikzcd}
 1
 \arrow[dd, "x"]
 \\
 \\
 a
 \end{tikzcd}
\]

我们希望在$1+0$出发的箭头和$1$出发的箭头之间建立一一对应关系。箭头$h$由对$(x, \mbox{!`})$决定。由于只有一个$\mbox{!`}$,$h$和$x$之间存在双射。

我们定义$\beta_a$将任何由对$(x, \mbox{!`})$定义的$h$映射到$x$。反过来,$\beta^{-1}_a$将$x$映射到对$(x, \mbox{!`})$。但它是一个自然变换吗?

为了回答这个问题,我们需要考虑当我们从$a$转移到通过箭头$g \colon a \to b$与之相连的某个$b$时会发生什么。我们现在有两个选择:
\begin{itemize}
\item 通过后复合$g$使$h$切换焦点。我们得到一个新的对$(y = g \circ x, \mbox{!`})$。然后用$\beta_b$跟随它。
\item 使用$\beta_a$将$(x, \mbox{!`})$映射到$x$。然后用后复合$(g \circ -)$跟随它。
\end{itemize}
在这两种情况下,我们得到相同的箭头$y = g \circ x$。因此,映射$\beta$是自然的。因此$1 + 0$同构于$1$。

在Haskell中,我们可以定义形成同构的两个函数,但无法直接表达它们是彼此逆的事实。
\begin{haskell}
f :: Either () Void -> ()
f (Left ()) = ()
f (Right _) = ()

f_1 :: () -> Either () Void
f_1 _ = Left ()
\end{haskell}
函数定义中的下划线通配符表示参数被忽略。\hask{f}定义中的第二个子句是多余的,因为没有类型\hask{Void}的项。

\subsection{某物加零}

一个非常类似的论证可以用来证明$a + 0 \cong a$。下图解释了这一点。
\[
 \begin{tikzcd}
 a
 \arrow[dr,  bend left, "\text{Left}"']
 \arrow[ddr, bend right, "f"']
 && 0
 \arrow[dl, bend right, "\text{Right}"]
 \arrow[ddl, bend left, "\mbox{!`}"]
 \\
&a + 0
\arrow[d, dashed, "h"]
\\
& x
 \end{tikzcd}
 \qquad
 \begin{tikzcd}
 a
 \arrow[dd, "f"]
 \\
 \\
 x
 \end{tikzcd}
\]

我们可以通过实现一个(多态的)函数\hask{h}来将这个论证翻译到Haskell中,该函数适用于任何类型\hask{a}。

\begin{exercise}
在Haskell中实现形成\hask{(Either a Void)}和\hask{a}之间同构的两个函数。
\end{exercise}

我们可以使用类似的论证来证明$0 + a \cong a$,但和类型的一个更一般的性质使得这变得不必要。
\subsection{交换性}

在定义和类型的图中存在一个很好的左右对称性,这表明它满足交换律,$a + b \cong b + a$。

让我们考虑从这个公式的两边映射出来。你可以很容易地看到,对于每一个由左边的对$(f, g)$决定的$h$,右边都有一个对应的$h'$由对$(g, f)$给出。这建立了箭头的双射。

\[
 \begin{tikzcd}
 a
 \arrow[dr,  bend left, "\text{Left}"']
 \arrow[ddr, bend right, "f"']
 && b
 \arrow[dl, bend right, "\text{Right}"]
 \arrow[ddl, bend left, "g"]
 \\
&a + b
\arrow[d, dashed, "h"]
\\
& x
 \end{tikzcd}
 \qquad
 \begin{tikzcd}
 b
 \arrow[dr,  bend left, "\text{Left}"']
 \arrow[ddr, bend right, "g"']
 && a
 \arrow[dl, bend right, "\text{Right}"]
 \arrow[ddl, bend left, "f"]
 \\
&b + a
\arrow[d, dashed, "h'"]
\\
& x
 \end{tikzcd}
\]

\begin{exercise}
证明上述定义的双射是自然的。提示:$f$和$g$都通过后复合$k \colon x \to y$改变焦点。
\end{exercise}
\begin{exercise}
在Haskell中实现见证\hask{(Either a b)}和\hask{(Either b a)}之间同构的函数。注意这个函数是它自己的逆。
\end{exercise}

\subsection{结合性}

就像在算术中一样,我们定义的和是结合的:
\[(a + b) + c \cong a + (b + c) \]
很容易写出左边的映射:
\begin{haskell}
h :: Either (Either a b) c -> x
h (Left (Left a))  = f1 a
h (Left (Right b)) = f2 b
h (Right c)        = f3 c
\end{haskell}
注意使用了嵌套模式如\hask{(Left (Left a))}等。映射完全由三个函数定义。同样的函数可以用来定义右边的映射:
\begin{haskell}
h' :: Either a (Either b c) -> x
h' (Left a)          = f1 a
h' (Right (Left b))  = f2 b
h' (Right (Right c)) = f3 c
\end{haskell}
这建立了定义两个映射出的三函数之间的一一对应关系。这个映射是自然的,因为所有的焦点变化都是通过后复合完成的。因此,两边是同构的。

这段代码也可以用图的形式展示。以下是同构左边的图:
\[
 \begin{tikzcd}
 a
 \arrow[rd, "L"']
 \arrow[rrddd, bend right, red, "f_1"']
 && b
 \arrow[ld, "R"]
 \arrow[ddd, bend left=60, red, "f_2"]
&&c
 \arrow[lldd, "R"]
 \arrow[llddd, bend left, red, "f_3"]
 \\
 & a + b
  \arrow[rd, "L"']
 \\
&&(a + b) + c
 \arrow[d, dashed, "h"]
\\
&& x
 \end{tikzcd}
\]
\subsection{函子性}
由于和是通过映射出的性质定义的,很容易看出当我们改变焦点时会发生什么:它会“自然地”随着定义积的箭头的焦点变化。但当我们移动这些箭头的源时会发生什么?

假设我们有箭头将$a$和$b$映射到某个$a'$和$b'$:
\begin{align*}f &\colon a \to a' \\
g &\colon b \to b' 
\end{align*}
这些箭头与构造函数$\text{Left}$和$\text{Right}$的复合可以用来定义和之间的映射:
\[
 \begin{tikzcd}
 a
 \arrow[d, "f"]
 \arrow[dr,  bend left, "\text{Left}"']
  && b
 \arrow[d, "g"]
 \arrow[dl, bend right, "\text{Right}"]
 \\
 a'
 \arrow[rd, "\text{Left}"']
&a + b
\arrow[d, dashed, "h"]
& b'
\arrow[ld, "\text{Right}"]
\\
& a' + b'
 \end{tikzcd}
\]
箭头对$(\text{Left} \circ f, \text{Right} \circ g)$唯一地定义了箭头$h \colon a + b \to a' + b'$。这个箭头的符号是:
\[ a + b \xrightarrow{\langle f, g \rangle} a' + b' \]

这种将一对箭头提升到和上的性质称为和的\emph{函子性}。你可以想象它允许你转换和\emph{内部}的两个对象并得到一个新的和。
\begin{exercise}
证明函子性保持复合。提示:取两个可复合的箭头,$g \colon b \to b'$和$g' \colon b' \to b''$,并证明应用$g' \circ g$与先应用$g$将$a + b$转换为$a + b'$,然后应用$g'$将$a + b'$转换为$a + b''$得到相同的结果。
\end{exercise}

\begin{exercise}
证明函子性保持恒等。提示:使用$id_b$并证明它被映射到$id_{a+b}$。
\end{exercise}

\subsection{对称幺半群范畴}
当一个孩子学习加法时,我们称之为算术。当一个成年人学习加法时,我们称之为余积范畴。

无论我们是加数字、组合箭头,还是构造对象的和,我们都在重复将复杂事物分解为更简单组成部分的相同思想。

正如老子所说,当事物结合在一起形成新事物,并且操作是结合的,并且它有一个中性元素时,我们就知道如何处理万物。

我们定义的和类型满足以下性质:
\begin{align*}
a + 0 &\cong a \\
a + b &\cong b + a \\
(a + b) + c &\cong a + (b + c)
\end{align*}
并且它是函子的。具有这种操作类型的范畴称为\emph{对称幺半群范畴}。当操作是和(余积)时,它被称为\emph{余积范畴}。在下一章中,我们将看到另一种称为\emph{积范畴}的幺半群结构,没有“余”字。
\end{document}
